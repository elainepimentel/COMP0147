\documentclass[11pt,a4paper]{article}

%====================================================================
% LaTeX Packages
%====================================================================
\usepackage{geometry}
\usepackage{enumerate}
\usepackage{mdwlist} % Allows for suspeding and resuming lists.
\geometry{a4paper,left=1.5cm,right=1.5cm,top=2.5cm,bottom=2cm}
\usepackage{epsf}
\usepackage{amsmath,amssymb}
\usepackage{array} % Allows fixing row width.
\usepackage{amsthm} % Allows for \qed symbol
\usepackage{graphicx}
\usepackage{nicefrac}
\usepackage{multicol}
\usepackage{xspace} % Introduces spaces after macros only when needed.
%--------------------------------------------------------------------


% Information about the course.
\newcommand{\strcourse}{Discrete Mathematics - COMP0147}
\newcommand{\strinstitution}{\includegraphics[height=1cm]{UCL-logo.jpeg}}
\newcommand{\strsemester}{2025 - Term 1}

% Information about the instructor.
\newcommand{\strinstructor}{Prof. Elaine Pimentel}
\newcommand{\stremail}{e.pimentel@ucl.ac.uk}

% Information about assigment
\newcommand{\strtypeofassignment}{List of exercises}
\newcommand{\strsolution}{(Solution)}
\newcommand{\strinstructorsolution}{Solution}
\newcommand{\streasy}{\level{Easy}}
\newcommand{\strmedium}{\level{Medium}}
\newcommand{\strhard}{\level{Hard}}
\newcommand{\strnoteondifficultylevel}{
\paragraph{Note:} 
The exercises are classified into difficulty levels:
easy, medium, and hard.
This classification, however, is only indicative.
Different people may disagree about the difficulty level of the same exercise.
Do not be discouraged if you see a difficult exercise--you may find that it is actually easy, by discovering a simpler way to solve it!
}
%--------------------------------------------------------------------

\newcommand{\level}[1]{\textbf{\texttt{[#1]}}\xspace}
\newcommand{\qm}[1]{``#1''}
\newcommand{\prop}[1]{\emph{\qm{#1}}} % Proposition in natural language
\newcommand{\lazyfrac}[2]{#1/#2}
\newcommand{\crossline}{\noindent\makebox[\linewidth]{\rule{\textwidth}{1pt}}}
\newcommand{\ie}{{\em i.e.}}
\newcommand{\eg}{{\em e.g.}}

\newcommand{\imp}{\rightarrow} % Implication symbol
\newcommand{\doubleimp}{\leftrightarrow} % Double implication symbol

\newcommand{\floor}[1]{\left\lfloor #1 \right\rfloor}
\newcommand{\ceil}[1]{\left\lceil #1 \right\rceil}

\newcommand{\green}[1]{{\color{green} #1}}
\newcommand{\red}[1]{{\color{red} #1}}


\usepackage{pifont}
\newcommand{\cmark}{\ding{51}}
\newcommand{\xmark}{\ding{55}}

\newcommand{\yes}{\green{\cmark}}
\newcommand{\no}{\red{\xmark}}


\usepackage{tikz}
\newcommand*\circled[1]{\tikz[baseline=(char.base)]{
            \node[shape=circle,draw,inner sep=2pt] (char) {#1};}}
\newcommand{\orgcellA}[3]{\begin{Large}$({#1},{#2})$\end{Large} {\circled{#3}}}
\newcommand{\orgcellB}[2]{\begin{Large}$({#1},{#2})$\end{Large} {$\ldots$}}

% We need this in order to be able to print both a handouts version
% and a solution version of the homework.
\ifdefined\hidesolution
	\newcommand{\solution}[1]{}  
\else
	\newcommand{\solution}[1]{\paragraph{\strinstructorsolution:} #1 \vspace{4mm}}
\fi

\newcommand{\homeworktitle}[2]{
\begin{center}
\begin{flushleft}
\noindent \textbf{\strinstitution \hfill \strsemester} \\
\textbf{\strcourse \hfill \strinstructor}
\end{flushleft} 
\ \\
\textbf{\MakeUppercase{\strtypeofassignment}}\\
\textsc{#1\\
(#2)}
\end{center}
}

\newcommand{\noteondifficultylevel}{
\strnoteondifficultylevel
}
%--------------------------------------------------------------------







%====================================================================
% Commands particular to this file
%====================================================================

%--------------------------------------------------------------------

\begin{document}

%====================================================================
\homeworktitle{Sets, functions, sequences, sums, cardinality}{Rosen - Chapter 2}

\crossline

\paragraph{Required reading for this list:}
\emph{Discrete Mathematics and Its Applications} (Rosen, 7\textsuperscript{th} Edition):
\begin{itemize}
\item Chapter 2.1: \emph{Sets}
\item Chapter 2.2: \emph{Set Operations}
\item Chapter 2.3: \emph{Functions}
\item Chapter 2.4: \emph{Sequences and Summations}
\item Chapter 2.5: \emph{Cardinality of Sets}
\end{itemize}

\noteondifficultylevel

\crossline


\begin{enumerate}


\item (Rosen 2.1-5) Determine whether each of the following pairs of sets are equal or not.

\begin{enumerate}
\item \streasy $\{1,3,3,3,5,5,5,5,5\}$ and $\{5,3,1\}$



\item \streasy $\{\{1\}\}$ and $\{1,\{1\}\}$



\item \streasy $\varnothing$ and $\{\varnothing\}$


\end{enumerate}

\item (Rosen 2.1-9) Determine whether each of the following statements is true or false.

\begin{multicols}{3}
\begin{enumerate}
\item \streasy $0 \in \varnothing$ 

\item \streasy $\varnothing \in \{0\}$

\item \streasy $\{0\} \subset \varnothing$

\item \streasy $\varnothing \subset \{0\} $

\item \streasy $\{0\} \in \{0\} $

\item \streasy $\{0\} \subset \{0\}$

\item \streasy $\{0\} \subseteq \{0\}$
\end{enumerate}
\end{multicols}



\item (Rosen 2.1-11) Determine whether each of the following statements is true or false.

\begin{multicols}{3}
\begin{enumerate}
\item \streasy $x \in \{x\}$ 


\item \streasy $\{x\} \subseteq \{x\}$ 


\item \streasy $\{x\} \in \{x\}$ 

\item \streasy $\{x\} \subseteq \{\{x\}\}$ 


\item \streasy $\varnothing \subseteq \{x\}$ 


\item \streasy $\varnothing \in \{x\}$ 

\end{enumerate}
\end{multicols}

\item (Rosen 2.1-21) Find the power set of the following sets, where $a$ and $b$ are distinct elements.

\begin{multicols}{3}
\begin{enumerate}
\item \streasy $\{a\}$ 

\item \streasy $\{a,b\}$ 
\solution{}

\item \streasy $\{\varnothing,\{\varnothing\}\}$ 
\end{enumerate}
\end{multicols}



\item (Rosen 2.1-27) Let $A = \{a,b,c,d\}$ and $B = \{x,y\}$. Find

\begin{multicols}{2}
\begin{enumerate}
\item \streasy $A \times B$

\item \streasy $B \times A$
\end{enumerate}
\end{multicols}



\item \streasy (Rosen 2.1-30) Suppose $A \times B = \varnothing$, where $A$ and $B$ are sets. What can you conclude?


\item \streasy (Rosen 2.1-39) Explain why $A \times B \times C$ and $(A \times B) \times C$ are not the same set.



\item \strhard (Rosen 2.1-45) 
The defining property of an ordered pair is that two ordered pairs are equal if and only if their first elements are equal and their second elements are equal. Surprisingly, instead of taking the ordered pair as a primitive concept, we can construct ordered pairs using basic notions from set theory. Show that if we define the ordered pair $(a,b)$ to be $\{\{a\},\{a,b\}\}$, then $(a,b) = (c,d)$ if and only if $a = c$ and $b = d$. [Hint: First show that $\{\{a\},\{a,b\}\} = \{\{c\},\{c,d\}\}$ if and only if $a = c$ and $b = d$.]


\item (Rosen 2.2-15) Show that if $A$ and $B$ are sets, then $\overline{A \cup B} = \overline{A} \cap \overline{B}$:
(This is one of De Morgan's laws.)

\begin{enumerate}
\item \strmedium By showing that each side is a subset of the other.


\item \streasy Using a membership table.

\end{enumerate}

\item (Rosen 2.2-16) Let $A$ and $B$ be sets. Show that:
\begin{enumerate}
\item \streasy $(A\cap B)\subseteq A$


\item \streasy $A-B \subseteq A$


\item \streasy $A\cup (B-A)=(A\cup B)$

\end{enumerate}

\item (Rosen 2.2-18) Let $A$, $B$, and $C$ be sets. Using logical connective manipulation, show that:
\begin{enumerate}
\item \strmedium $(A\cup B)\subseteq (A\cup B \cup C)$


\item \strmedium $(A-B)-C\subseteq A-C$


\item \strmedium $(B-A)\cup (C-A)=(B\cup C)-A$


\end{enumerate}

\item (Rosen 2.2-29) What can you say about sets $A$ and $B$ if you know that:

\begin{multicols}{3}
\begin{enumerate}
\item $A \cup B = A$?

\item $A \cap B = A$?

\item $A - B = A$?

\item $A \cap B = B \cap A$?

\item $A - B = B - A$?
\end{enumerate}
\end{multicols}


\item (Rosen 2.2-50) Determine $\bigcup_{i=1}^{\infty} A_i$ and $\bigcap_{i=1}^{\infty} A_i$ for each $A_i$ below:
\begin{enumerate}
\item \strmedium $A_i=\{i,i+1,i+2,...\}$.

\item \strmedium $A_i=\{0,i\}$.


\item \strmedium $A_i=(0,i)$, i.e., the set $\{x \in \mathbb{R} \mid 0<x<1\}$.


\item \strmedium $A_i=(i,\infty)$, i.e., the set $\{x \in \mathbb{R} \mid x > i \}$.

\end{enumerate}

\item (Rosen 2.3-1) Why is $f$ not a function from $\mathbb{R}$ to $\mathbb{R}$ if

\begin{enumerate}
\item \streasy $f(x) = 1 / x$?

\item \streasy $f(x) = \sqrt{x}$?

\item \streasy $f(x) = \pm \sqrt{(x^{2}+1)}$?

\end{enumerate}

\item (Rosen 2.3-4) Find the domain and range 
of the functions below. 
Note that, in each case, to find the domain you should
identify the set of elements to which
the function assigns a value.

\begin{enumerate}
\item \streasy the function that assigns to each non-negative integer
its last digit;

\item \streasy the function that assigns the next largest integer
to a positive integer;

\item \streasy the function that assigns to a binary string
the number of $1$ bits in that string;

\item \streasy the function that assigns to a binary string
the total number of bits in that string.

\end{enumerate}

\item (Rosen 2.3-9) Find the value of:

\begin{multicols}{2}
\begin{enumerate}
\item \streasy $\ceil{\frac{3}{4}}$

\item \streasy $\floor{\frac{7}{8}}$

\item \streasy $\ceil{-\frac{3}{4}}$

\item \streasy $\floor{-\frac{7}{8}}$

\item \streasy  $\ceil{3}$

\item \streasy $\floor{-1}$

\item \streasy $\ceil{\frac{1}{2} + \ceil{\frac{3}{2}}}$

\item \streasy $\floor{\frac{1}{2} \cdot \floor{\frac{5}{2}}}$
\end{enumerate}
\end{multicols}

\item (Rosen 2.3-12) Determine which of the following functions
from $\mathbb{Z}$ to $\mathbb{Z}$ are injective,
surjective, and bijective.
\begin{enumerate}
\item \streasy $f(n) = n-1$

\item \streasy $f(n) = n^{2} + 1$

\item \streasy $f(n) = n^{3}$

\item \streasy $f(n) = \lceil n/2 \rceil$

\end{enumerate}

\item (Rosen 2.3-33) Let $g$ be a function from $A$ to $B$
and $f$ a function from $B$ to $C$.

\begin{enumerate}
\item \streasy Show that if $f$ and $g$ are both injective,
then $f \circ g$ is also injective.

\item \streasy Show that if $f$ and $g$ are both surjective,
then $f \circ g$ is also surjective.

\end{enumerate}

\item (Rosen 2.3-42) Let $f$ be a function from $\mathbb{R}$
to $\mathbb{R}$ defined as $f(x) = x^{2}$.
Find
\begin{enumerate}
\item \streasy $f^{-1}(\{ 1 \})$

\item \strmedium $f^{-1}(\{ x \mid 0 < x < 1 \})$

\item \strmedium $f^{-1}(\{ x \mid x > 4 \})$

\end{enumerate}

\item \strhard (Rosen 2.3-54) Prove that if $x$ is a real number, 
then $\lfloor -x \rfloor=-\lceil x \rceil$ and 
$\lceil -x \rceil = - \lfloor x \rfloor$.

\item (Rosen 2.3-70) Prove or disprove each of the following statements:
\begin{enumerate}
\item $\lfloor \lceil x \rceil \rfloor = \lceil x \rceil \quad \forall x \in \mathbb{R}$.
\item $\lfloor x+y \rfloor = \lfloor x \rfloor +  \lfloor y \rfloor \quad \forall x,y\in \mathbb{R}$.
\item $\lceil \lceil \frac{x}{2} \rceil / 2 \rceil = \lceil \frac{x}{4} \rceil \quad \forall x \in \mathbb{R}$.
\item $\lfloor \sqrt{\lceil x \rceil} \rfloor=\lfloor \sqrt{x} \rfloor \quad \forall x \in \mathbb{R}^+$.
\item $\lfloor x \rfloor + \lfloor y \rfloor + \lfloor x+y \rfloor \leq \lfloor 2x \rfloor + \lfloor 2y \rfloor \quad \forall x,y \in \mathbb{R}$.
\end{enumerate}

\item (Rosen 2.3-79)
\begin{enumerate}
\item[a)] \streasy Show that if a set $S$ has cardinality $m$, where $m$ is a positive integer, then there is a one-to-one correspondence between $S$ and the set $\{1,2,\ldots,m\}$.

\item[b)] \strmedium Show that if $S$ and $T$ are two sets each with $m$ elements, where $m$ is a positive integer, then there is a one-to-one correspondence between $S$ and $T$.
\end{enumerate}

\item (Rosen 2.4-3) What are the terms $a_{0}, a_{1}, a_{2}$ and $a_{3}$ 
of the sequence $\{a_{n}\}$ where $a_{n}$ is given by

\begin{multicols}{2}
\begin{enumerate}
\item \streasy $2^{n}+1$

\item \streasy $(n+1)^{n+1}$

\item \streasy $\floor{\frac{n}{2}}$

\item \streasy $\floor{\frac{n}{2}} + \ceil{\frac{n}{2}}$
\end{enumerate}
\end{multicols}

\item (Rosen 2.4-5) List the first 10 terms of these
sequences.

\begin{enumerate}
\item \streasy the sequence starting with 2 where each term
is 3 more than the previous one;

\item \streasy the sequence listing each positive integer
three times in increasing order;

\item \streasy the sequence listing all positive odd integers
in increasing order, listing each odd twice;

\item \streasy the sequence whose $n$-th term is $n!-2^{n}$;

\item \streasy the sequence starting with 3 where each
subsequent term is twice the previous term;


\item \streasy the sequence whose first term is 2, the second is 4,
and each next term is the sum of the previous two terms;

\item \streasy the sequence whose $n$-th term is the number
of bits in the binary representation of $n$;


\item \streasy the sequence whose $n$-th term is the number
of letters in the Portuguese word for $n$.

\end{enumerate}

\item (Rosen 2.4-26) For each of the integer lists below, 
give a simple formula or rule that generates the terms of a sequence of 
integers that starts with the given list. 
Assuming your formula is correct, give the next three elements 
of the sequence.
\begin{enumerate}
\item \streasy $3,6,11,18,27,38,51,66,83,102,...$

\item \streasy $7,11,15,19,23,27,31,35,39,43,...$

\item \streasy $1,10,11,100,101,110,111,1000,1001,1010,1011,...$

\item \streasy $1,2,2,2,3,3,3,3,3,5,5,5,5,5,5,5,...$


\item \strhard $0,2,8,26,80,242,728,2\,186,6\,560,19\,682,...$

\item \strhard $1,3,15,105,945,10\,395,135\,135,2\,027\,025,34\,459\,425,...$


\item \strmedium $1,0,0,1,1,1,0,0,0,0,1,1,1,1,1,...$

\end{enumerate} 

\item (Rosen 2.4-43) What are the values of the following products:
\begin{enumerate}
\item \streasy $\prod_{i=0}^{13}i$


\item \streasy $\prod_{i=5}^{10}i$


\item \strmedium $\prod_{i=0}^{99}(-1)^{i}$


\item \strmedium $\prod_{i=0}^{11}2$
\end{enumerate}

\item (Rosen 2.5-1) Determine if each of the sets below is 
finite, countably infinite, or uncountable.
For countably infinite sets, list the first 10 elements in an enumeration.

\begin{enumerate}
\item \streasy the negative integers


\item \streasy the even integers


\item \streasy the integers less than 100


\item \streasy the real numbers between $0$ and $\nicefrac{1}{2}$


\item \streasy the positive integers less than $1\,000\,000\,000$


\item \streasy the integers that are multiples of 7
\end{enumerate}

\item (Rosen 2.5-2) Determine if each of the sets below is countable or uncountable. 
For countable sets, list the first 10 elements in an enumeration.
\begin{enumerate}
\item \streasy the integers greater than 10;


\item \streasy the negative odd integers;

\item \streasy the real numbers between 0 and 2;


\item \streasy the integers that are multiples of 10.

\end{enumerate}

\item (Rosen 2.5-11) Give an example of two uncountable sets $A$ and $B$ such that $A \cap B$ is

\begin{enumerate}
\item \strmedium finite

\item \strmedium countably infinite

\item \strmedium uncountable
\end{enumerate}

\item (Rosen 2.5-34) Determine if each of the sets below is countable or not. 
For countable sets, list the first 10 elements in an enumeration.
\begin{enumerate}
\item \streasy the integers not divisible by 3.


\item \streasy the integers divisible by 5 but not by 7.


\item \strmedium the real numbers with decimal representation containing only 1s.


\item \strmedium the real numbers with decimal representation containing only 1s or 9s.

\end{enumerate}

\item (Rosen 2.5-28) \strhard Show that the set $\mathbb{Z}^{+} \times \mathbb{Z}^{+}$ is countable.


\end{enumerate}

\end{document}
