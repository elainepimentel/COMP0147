\input{hmwrk_header-sol.tex}

%====================================================================
% Commands particular to this file
%====================================================================

%--------------------------------------------------------------------

\begin{document}

%====================================================================
\homeworktitle{Sets, functions, sequences, sums, cardinality}{Rosen - Chapter 2}

\crossline

\paragraph{Required reading for this list:}
\emph{Discrete Mathematics and Its Applications} (Rosen, 7\textsuperscript{th} Edition):
\begin{itemize}
\item Chapter 2.1: \emph{Sets}
\item Chapter 2.2: \emph{Set Operations}
\item Chapter 2.3: \emph{Functions}
\item Chapter 2.4: \emph{Sequences and Summations}
\item Chapter 2.5: \emph{Cardinality of Sets}
\end{itemize}

\noteondifficultylevel

\crossline


\begin{enumerate}


\item (Rosen 2.1-5) Determine whether each of the following pairs of sets are equal or not.

\begin{enumerate}
\item \streasy $\{1,3,3,3,5,5,5,5,5\}$ and $\{5,3,1\}$

\solution{The sets are equal, as they contain the same elements.}

\item \streasy $\{\{1\}\}$ and $\{1,\{1\}\}$

\solution{The sets are different: the element $1$ belongs to the second set, but not the first.}

\item \streasy $\varnothing$ and $\{\varnothing\}$

\solution{The sets are different: the first is the empty set, while the second is the set whose only element is the empty set.}
\end{enumerate}

\item (Rosen 2.1-9) Determine whether each of the following statements is true or false.

\begin{multicols}{3}
\begin{enumerate}
\item \streasy $0 \in \varnothing$ 

\item \streasy $\varnothing \in \{0\}$

\item \streasy $\{0\} \subset \varnothing$

\item \streasy $\varnothing \subset \{0\} $

\item \streasy $\{0\} \in \{0\} $

\item \streasy $\{0\} \subset \{0\}$

\item \streasy $\{0\} \subseteq \{0\}$
\end{enumerate}
\end{multicols}

\solution{ a) False b) False c) False d) True e) False f) False g) True}

\item (Rosen 2.1-11) Determine whether each of the following statements is true or false.

\begin{multicols}{3}
\begin{enumerate}
\item \streasy $x \in \{x\}$ 
\solution{True.}

\item \streasy $\{x\} \subseteq \{x\}$ 
\solution{True.}

\item \streasy $\{x\} \in \{x\}$ 
\solution{False.}

\item \streasy $\{x\} \subseteq \{\{x\}\}$ 
\solution{True.}

\item \streasy $\varnothing \subseteq \{x\}$ 
\solution{True.}

\item \streasy $\varnothing \in \{x\}$ 
\solution{False.}
\end{enumerate}
\end{multicols}

\item (Rosen 2.1-21) Find the power set of the following sets, where $a$ and $b$ are distinct elements.

\begin{multicols}{3}
\begin{enumerate}
\item \streasy $\{a\}$ 

\item \streasy $\{a,b\}$ 
\solution{}

\item \streasy $\{\varnothing,\{\varnothing\}\}$ 
\end{enumerate}
\end{multicols}

\solution{
\begin{enumerate}
\item $\mathcal{P}(\{a\}) = \{ \varnothing, \{a\} \}$

\item $\mathcal{P}(\{a,b\}) = \{ \varnothing, \{a\}, \{b\}, \{a,b\} \}$

\item $\mathcal{P}(\{\varnothing,\{\varnothing\}\}) = \{\varnothing, \{\varnothing\}, \{\{\varnothing\}\}, \{\varnothing,\{\varnothing\}\} \}$
\end{enumerate}
}

\item (Rosen 2.1-27) Let $A = \{a,b,c,d\}$ and $B = \{x,y\}$. Find

\begin{multicols}{2}
\begin{enumerate}
\item \streasy $A \times B$

\item \streasy $B \times A$
\end{enumerate}
\end{multicols}

\solution{
\begin{enumerate}
\item  $A \times B = \{(a,x), (a,y), (b,x), (b,y), (c,x), (c,y), (d,x), (d,y) \}$

\item  $B \times A = \{(x,a), (x,b), (x,c), (x,d), (y,a), (y,b), (y,c), (y,d) \}$
\end{enumerate}
}

\item \streasy (Rosen 2.1-30) Suppose $A \times B = \varnothing$, where $A$ and $B$ are sets. What can you conclude?

\solution{
At least one of $A$ or $B$ is the empty set.
}

\item \streasy (Rosen 2.1-39) Explain why $A \times B \times C$ and $(A \times B) \times C$ are not the same set.

\solution{
The elements of $A \times B \times C$ are ordered triples $(a,b,c)$ such that $a \in A$, $b \in B$, and $c \in C$.

The elements of $(A \times B) \times C$ are ordered pairs $((a,b),c)$, where the first element is itself an ordered pair $(a,b)$ with $a \in A$ and $b \in B$, and the second element is $c \in C$.
}

\item \strhard (Rosen 2.1-45) 
The defining property of an ordered pair is that two ordered pairs are equal if and only if their first elements are equal and their second elements are equal. Surprisingly, instead of taking the ordered pair as a primitive concept, we can construct ordered pairs using basic notions from set theory. Show that if we define the ordered pair $(a,b)$ to be $\{\{a\},\{a,b\}\}$, then $(a,b) = (c,d)$ if and only if $a = c$ and $b = d$. [Hint: First show that $\{\{a\},\{a,b\}\} = \{\{c\},\{c,d\}\}$ if and only if $a = c$ and $b = d$.]
\solution{
First we prove the statement mentioned in the hint. The ``if'' part is immediate from the definition of equality. The ``only if'' part is rather subtle. We want to show that if $\{\{a\},\{a,b\}\} = \{\{c\},\{c,d\}\}$, then $a= c$ and $b = d$. First consider the case in which $a\not=  b$. Then $\{\{a\}, \{a, b\}\}$ has exactly two elements, both of which are sets; exactly one of them contains one element, and exactly one of them contains two elements. Thus $\{\{c\},\{c,d\}\}$ must have the same property; hence $c$ cannot equal $d$, and so $\{c\}$ is the element containing one element. Hence $\{a\} = \{c\}$, and so $a = c$. Also in this case the two-element elements $\{a, b\}$ and $\{c, d\}$ must be equal, and since $b \not= a = c$, we must have $b = d$. The other possibility is that $a = b$. Then $\{\{a\},\{a,b\}\} =\{\{a\}\}$, a set with one element. Hence $\{\{c\},\{c,d\}\}$ must also have only one element, which can only happen when $c = d$ and the set is $\{\{c\}\}$. It then follows that $a = c$, and hence $b= d$, as well.
Now there is really nothing else to prove. The property that we want ordered pairs to have is precisely the one that we just proved is satisfied by this definition. Furthermore, if we look at the proof, then it is clear how to ``recover'' both $a$ and $b$ from $\{\{a\}, \{a, b\}\}$. If this set has two elements, then $a$ is the unique element in the one-element element of this set, and $b$ is the unique member of the two-element element of this set other than $a$. If this set has only one element, then $a$ and $b$ are both equal to the unique element of the unique element of this set.
}


\item (Rosen 2.2-15) Show that if $A$ and $B$ are sets, then $\overline{A \cup B} = \overline{A} \cap \overline{B}$:
(This is one of De Morgan's laws.)

\begin{enumerate}
\item \strmedium By showing that each side is a subset of the other.

\solution{
We need to show that $\overline{A \cup B} \subseteq \overline{A} \cap \overline{B}$ and $\overline{A} \cap \overline{B} \subseteq \overline{A \cup B}$. This is equivalent to showing that, for every $x$ in the universe, $x \in \overline{A \cup B} \doubleimp x \in \overline{A} \cap \overline{B}$.

Thus, we can verify that:
\begin{align*}
x \in \overline{A \cup B}
&\doubleimp (x \not\in A \cup B) & \text{(definition of complement)} \\
&\doubleimp \neg{(x \in A \cup B)} & \text{(definition of $\not\in$)} \\
&\doubleimp \neg{(x \in A \vee x \in B)} & \text{(definition of union)} \\
&\doubleimp \neg{(x \in A)} \wedge \neg{(x \in B)} & \text{(De Morgan)} \\
&\doubleimp (x \not\in A) \wedge (x \not\in B) & \text{(definition of $\not\in$)} \\
&\doubleimp (x \in \overline{A}) \wedge (x \in \overline{B}) & \text{(definition of complement)} \\
&\doubleimp x \in \overline{A} \cap \overline{B} & \text{(definition of intersection)} \\
\end{align*}
}

\item \streasy Using a membership table.

\solution{
\renewcommand{\arraystretch}{1.5}
$$
\begin{array}{|c|c||c|c|c|c|c|}
\hline
A & B & A \cup B & \overline{(A \cup B)} & \overline{A} & \overline{B} & \overline{A} \cap \overline{B} \\
\hline \hline
1 & 1 & 1 & 0 & 0 & 0 & 0 \\ \hline
1 & 0 & 1 & 0 & 0 & 1 & 0 \\ \hline
0 & 1 & 1 & 0 & 1 & 0 & 0 \\ \hline
0 & 0 & 0 & 1 & 1 & 1 & 1 \\ \hline
\end{array}
$$
\renewcommand{\arraystretch}{1}
}
\end{enumerate}

\item (Rosen 2.2-16) Let $A$ and $B$ be sets. Show that:
\begin{enumerate}
\item \streasy $(A\cap B)\subseteq A$
\solution{We need to show that if $x \in A\cap B$, then $x \in A$.
\begin{align*}
x \in A \cap B 
 &\imp x \in A \wedge x \in B & \text{(definition of intersection)} \\
 &\imp x \in A & \text{(simplifying the conjunction)}
\end{align*}
}

\item \streasy $A-B \subseteq A$
\solution{
We need to show that if $x \in A-B$, then $x \in A$.
\begin{align*}
x \in A - B 
 &\imp x \in A \wedge \neg{(x \in B)} & \text{(definition of difference)} \\
 &\imp x \in A & \text{(simplifying the conjunction)}
\end{align*}
}

\item \streasy $A\cup (B-A)=(A\cup B)$
\solution{
We need to show that $x \in A\cup (B-A)$ if and only if $x \in A\cup B$.
\begin{align*}
A \cup (B-A) 
 &=  \{x  \mid  x \in A \vee x \in (B-A) \} &  \text{(definition of union)} \\
 &=  \{x  \mid  x \in A \vee (x \in B \wedge \neg{(x \in A)})  \} & \text{(definition of difference)} \\
 &= \{x  \mid  (x \in A \vee  x \in B) \wedge (x \in A \vee \neg{(x \in A)}) \} & \text{(distributivity)} \\
 &= \{x  \mid  (x \in A \vee  x \in B) \wedge \textit{true} \} & \text{(law of negation)} \\
 &= \{x  \mid  x \in A \vee  x \in B \} & \text{(identity law)} \\
 &= A \cup B & \text{(definition of union)}
\end{align*}
}
\end{enumerate}

\item (Rosen 2.2-18) Let $A$, $B$, and $C$ be sets. Using logical connective manipulation, show that:
\begin{enumerate}
\item \strmedium $(A\cup B)\subseteq (A\cup B \cup C)$
\solution{
We need to show that if $x \in A \cup B$, then $x \in A\cup B \cup C$.
\begin{align*}
x \in A \cup B 
 &\imp x \in A \vee x \in B & \text{(definition of union)} \\
 &\imp x \in A \vee x \in B \vee x \in C & \text{(disjunctive addition)} \\
 &\imp x \in A \cup B \cup C & \text{(definition of union)}
\end{align*}
}

\item \strmedium $(A-B)-C\subseteq A-C$
\solution{
We need to show that if $x \in (A-B)-C$, then $x \in A-C$.
\begin{align*}
x \in (A-B)-C &\imp x \in (A-B) \wedge \neg{(x \in C)} & \text{(definition of difference)} \\
&\imp (x \in A \wedge \neg{(x \in B)}) \wedge \neg{(x \in C)} & \text{(definition of difference)} \\
&\imp x \in A \wedge \neg{(x \in B)} \wedge \neg{(x \in C)} & \text{(associativity)} \\
&\imp x \in A \wedge \neg{(x \in C)} & \text{(simplifying conjunction)} \\
&\imp x \in A - C & \text{(definition of difference)}
\end{align*}
}

\item \strmedium $(B-A)\cup (C-A)=(B\cup C)-A$
\solution{
We need to show that $x \in (B-A) \cup (C-A)$ if and only if $x \in (B \cup C)-A$.
\begin{align*}
x \in ( (B-A) \cup (C-A) ) &\doubleimp  x \in (B-A) \vee x \in (C-A) &  \text{(definition of union)} \\
&\doubleimp  (x \in B \wedge \neg(x \in A)) \vee (x \in C \wedge \neg(x \in A)) &  \text{(definition of difference)} \\
&\doubleimp  (x \in B \vee x \in C) \wedge \neg(x \in A) & \text{(distributivity)} \\
&\doubleimp  (x \in B \cup C) \wedge \neg(x \in A) & \text{(definition of union)} \\
&\doubleimp x \in (B \cup C) - A & \text{(definition of difference)}
\end{align*}
}

\end{enumerate}

\item (Rosen 2.2-29) What can you say about sets $A$ and $B$ if you know that:

\begin{multicols}{3}
\begin{enumerate}
\item $A \cup B = A$?

\item $A \cap B = A$?

\item $A - B = A$?

\item $A \cap B = B \cap A$?

\item $A - B = B - A$?
\end{enumerate}
\end{multicols}

\solution{
\begin{enumerate}
\item $B \subseteq A$
\item $A \subseteq B$
\item $A \cap B = \varnothing$
\item Nothing, $A$ and $B$ can be any sets.
\item $A=B$
\end{enumerate}
}

\item (Rosen 2.2-50) Determine $\bigcup_{i=1}^{\infty} A_i$ and $\bigcap_{i=1}^{\infty} A_i$ for each $A_i$ below:
\begin{enumerate}
\item \strmedium $A_i=\{i,i+1,i+2,...\}$.
\solution{
\begin{align*}
\bigcup_{i=1}^{\infty} \{i,i+1,i+2,...\} &= \mathbb{N} - \{0\} \\
\bigcap_{i=1}^{\infty} \{i,i+1,i+2,...\} &= \varnothing
\end{align*}
}

\item \strmedium $A_i=\{0,i\}$.
\solution{
\begin{align*}
\bigcup_{i=1}^{\infty} \{0,i\} &= \mathbb{N} \\
\bigcap_{i=1}^{\infty} \{0,i\} &= \{0\}
\end{align*}
}

\item \strmedium $A_i=(0,i)$, i.e., the set $\{x \in \mathbb{R} \mid 0<x<1\}$.
\solution{
\begin{align*}
\bigcup_{i=1}^{\infty} (0,i) &= (0,\infty) \\
\bigcap_{i=1}^{\infty} (0,i) &= (0,1)
\end{align*}
}

\item \strmedium $A_i=(i,\infty)$, i.e., the set $\{x \in \mathbb{R} \mid x > i \}$.
\solution{
\begin{align*}
\bigcup_{i=1}^{\infty} (i,\infty) &= (1,\infty) \\
\bigcap_{i=1}^{\infty} (i,\infty) &= \varnothing
\end{align*}
}
\end{enumerate}

\item (Rosen 2.3-1) Why is $f$ not a function from $\mathbb{R}$ to $\mathbb{R}$ if

\begin{enumerate}
\item \streasy $f(x) = 1 / x$?
\solution{
$f(x)$ is not defined for all elements of the domain, since
$f(0) = 1/0$ is undefined.
}
\item \streasy $f(x) = \sqrt{x}$?
\solution{
$f(x)$ is not defined for all elements of the domain, 
since for $x < 0$ the function has no real output.
}
\item \streasy $f(x) = \pm \sqrt{(x^{2}+1)}$?
\solution{
$f(x)$ associates more than one element of the codomain
with the same element of the domain.
}
\end{enumerate}

\item (Rosen 2.3-4) Find the domain and range 
of the functions below. 
Note that, in each case, to find the domain you should
identify the set of elements to which
the function assigns a value.

\begin{enumerate}
\item \streasy the function that assigns to each non-negative integer
its last digit;
\solution{
Function: $\mathbb{Z}^{+} \cup \{0\} \rightarrow \{0,1,2,3,4,5,6,7,8,9\}$.
}
\item \streasy the function that assigns the next largest integer
to a positive integer;
\solution{
Function: $\mathbb{Z}^{+} \rightarrow \mathbb{Z}^{+} \backslash \{1\}$.
}
\item \streasy the function that assigns to a binary string
the number of $1$ bits in that string;
\solution{
Function: $\{0,1\}^{*} \rightarrow \mathbb{N}$
}
\item \streasy the function that assigns to a binary string
the total number of bits in that string.
\solution{
Function: $\{0,1\}^{*} \rightarrow \mathbb{N}$
}
\end{enumerate}

\item (Rosen 2.3-9) Find the value of:

\begin{multicols}{2}
\begin{enumerate}
\item \streasy $\ceil{\frac{3}{4}}$

\item \streasy $\floor{\frac{7}{8}}$

\item \streasy $\ceil{-\frac{3}{4}}$

\item \streasy $\floor{-\frac{7}{8}}$

\item \streasy  $\ceil{3}$

\item \streasy $\floor{-1}$

\item \streasy $\ceil{\frac{1}{2} + \ceil{\frac{3}{2}}}$

\item \streasy $\floor{\frac{1}{2} \cdot \floor{\frac{5}{2}}}$
\end{enumerate}
\end{multicols}

\solution{
\begin{multicols}{2}
\begin{enumerate}
\item $\ceil{\frac{3}{4}} = 1$

\item $\floor{\frac{7}{8}} = 0$

\item $\ceil{-\frac{3}{4}} = 0$

\item $\floor{-\frac{7}{8}} = -1$

\item $\ceil{3} = 3$

\item $\floor{-1} = -1$

\item 
$
  \ceil{\frac{1}{2} + \ceil{\frac{3}{2}}}
= \ceil{\frac{1}{2} + 2} 
= \ceil{\frac{5}{2}} 
= 3 
$

\item  
$
  \floor{\frac{1}{2} \cdot \floor{\frac{5}{2}}}
= \floor{\frac{1}{2} \cdot 2}
= \floor{1}
= 1
$
\end{enumerate}
\end{multicols}
}

\item (Rosen 2.3-12) Determine which of the following functions
from $\mathbb{Z}$ to $\mathbb{Z}$ are injective,
surjective, and bijective.
\begin{enumerate}
\item \streasy $f(n) = n-1$
\solution{
The function is injective, surjective, and bijective.
}
\item \streasy $f(n) = n^{2} + 1$
\solution{
The function is neither injective, nor surjective, nor bijective.
}
\item \streasy $f(n) = n^{3}$
\solution{
The function is injective, but not surjective, nor bijective.
}
\item \streasy $f(n) = \lceil n/2 \rceil$
\solution{
The function is not injective, is surjective, and not bijective.
}
\end{enumerate}

\item (Rosen 2.3-33) Let $g$ be a function from $A$ to $B$
and $f$ a function from $B$ to $C$.

\begin{enumerate}
\item \streasy Show that if $f$ and $g$ are both injective,
then $f \circ g$ is also injective.
\solution{Assume that both $f$ and $g$ are one-to-one. We need to show that $f \circ g$ is one-to-one. This means that we need to show that if $x$ and $y$ are two distinct elements of $A$, then $f(g(x)) \not= f(g(y))$. First, since $g$ is one-to-one, the definition tells us that $g(x) \not= g(y)$. Second, since now $g(x)$ and $g(y)$ are distinct elements of $B$, and since $f$ is one-to-one, we conclude that $f(g(x)) \not=f(g(y))$, as desired.}

\item \streasy Show that if $f$ and $g$ are both surjective,
then $f \circ g$ is also surjective.
\solution{Assume that both $f$ and $g$ are onto. We need to show that $f \circ g$ is onto. This means that we need to show that if $z$ is any element of $C$, then there is some element $x \in A$ such that $f(g(x)) = z$. First, since $f$ is onto, we can conclude that there is an element $y\in  B$ such that $f (y) = z$. Second, since g is onto and $y \in B$, we can conclude that there is an element $x \in A$ such that $g(x) = y$. Putting these together, we have $z = f(y) = f(g(x))$, as desired.
}
\end{enumerate}

\item (Rosen 2.3-42) Let $f$ be a function from $\mathbb{R}$
to $\mathbb{R}$ defined as $f(x) = x^{2}$.
Find
\begin{enumerate}
\item \streasy $f^{-1}(\{ 1 \})$
\solution{
$f^{-1}(\{ 1 \}) = \{-1,1\}$
}
\item \strmedium $f^{-1}(\{ x \mid 0 < x < 1 \})$
\solution{
$f^{-1}(\{ x \mid 0 < x < 1 \}) = \{ x \mid -1 < x < 1 \}$
}
\item \strmedium $f^{-1}(\{ x \mid x > 4 \})$
\solution{
$f^{-1}(\{ x \mid x > 4 \}) = \{x \mid x < -2 \vee x > 2\}$
}
\end{enumerate}

\item \strhard (Rosen 2.3-54) Prove that if $x$ is a real number, 
then $\lfloor -x \rfloor=-\lceil x \rceil$ and 
$\lceil -x \rceil = - \lfloor x \rfloor$.
\solution{
Let $x$ be written as $n + \epsilon$, where $n$
is an integer and $0 \leq \epsilon < 1$. 
There are two cases to consider, depending on whether $\epsilon = 0$ or not.

\paragraph{Case 1: $\epsilon = 0$.} 
Then $x = n$ and $\lfloor -x \rfloor= - \lceil x \rceil = -n$ and 
$\lceil -x \rceil = - \lfloor x \rfloor = -n$.

\paragraph{Case 2: $0 < \epsilon < 1$.} 
First, note that
\begin{align*}
\lfloor -x \rfloor &= \lfloor -(n + \epsilon) \rfloor \\
				   &= \lfloor -n -\epsilon \rfloor \\
				   &= -(n+1),
\end{align*}
and that
\begin{align*}
-\lceil x \rceil &= -\lceil n+1 \rceil \\
				 &= -(n+1),
\end{align*}
so $\lfloor -x \rfloor=-\lceil x\rceil$. 

Second, note that
\begin{align*}
\lceil -x \rceil &= \lceil -(n+\epsilon) \rceil \\
				   &= \lceil -n -\epsilon \rceil \\
				   &= -n,
\end{align*}
and that
\begin{align*}
- \lfloor x \rfloor &= - \lfloor n+\epsilon \rfloor \\
				 &= -n,
\end{align*}
so $\lceil -x \rceil = - \lfloor x \rfloor$.
}

\item (Rosen 2.3-70) Prove or disprove each of the following statements:
\begin{enumerate}
\item $\lfloor \lceil x \rceil \rfloor = \lceil x \rceil \quad \forall x \in \mathbb{R}$.
\item $\lfloor x+y \rfloor = \lfloor x \rfloor +  \lfloor y \rfloor \quad \forall x,y\in \mathbb{R}$.
\item $\lceil \lceil \frac{x}{2} \rceil / 2 \rceil = \lceil \frac{x}{4} \rceil \quad \forall x \in \mathbb{R}$.
\item $\lfloor \sqrt{\lceil x \rceil} \rfloor=\lfloor \sqrt{x} \rfloor \quad \forall x \in \mathbb{R}^+$.
\item $\lfloor x \rfloor + \lfloor y \rfloor + \lfloor x+y \rfloor \leq \lfloor 2x \rfloor + \lfloor 2y \rfloor \quad \forall x,y \in \mathbb{R}$.
\end{enumerate}

\solution{
\begin{enumerate}
\item \textbf{True.} Since $\lceil x\rceil$ is an integer for every real $x$, taking the floor of an integer returns the same integer. More precisely, let $n=\lceil x\rceil\in\mathbb{Z}$. Then $\lfloor n\rfloor=n$, so $\lfloor\lceil x\rceil\rfloor=\lceil x\rceil$.
\item \textbf{False.} Counterexample: take $x=y=\tfrac12$. Then $\lfloor x+y\rfloor=\lfloor 1\rfloor=1$ while $\lfloor x\rfloor+\lfloor y\rfloor=0+0=0$, so equality fails. (In fact one always has $\lfloor x\rfloor+\lfloor y\rfloor\le\lfloor x+y\rfloor\le\lfloor x\rfloor+\lfloor y\rfloor+1$.)
\item \textbf{True.} Put $a=\frac{x}{2}$. We must show $\lceil \lceil a\rceil/2\rceil=\lceil a/2\rceil$ for every real $a$. Let $m=\lceil a\rceil\in\mathbb{Z}$. Then $m-1<a\le m$, so
\[
\frac{m-1}{2}<\frac{a}{2}\leq\frac{m}{2} 
\]
If $m=2k$ is even then $\frac{m-1}{2}=k-\tfrac12<\frac{a}{2}\le k$, hence $\lceil a/2\rceil=k=\lceil m/2\rceil$. If $m=2k+1$ is odd then $k<\frac{a}{2}\le k+\tfrac12$, so $\lceil a/2\rceil=k+1=\lceil m/2\rceil$. Thus in both cases $\lceil a/2\rceil=\lceil m/2\rceil=\lceil\lceil a\rceil/2\rceil$, which gives the desired equality with $a=x/2$.
\item \textbf{False.} Counterexample: take $x=3.9$. Then $\lceil x\rceil=4$, so $\lfloor\sqrt{\lceil x\rceil}\rfloor=\lfloor\sqrt{4}\rfloor=2$, whereas $\lfloor\sqrt{x}\rfloor=\lfloor\sqrt{3.9}\rfloor=1$. Hence equality fails in general.
\item \textbf{True.} Write $x=n+u$ and $y=m+v$ with $n=\lfloor x\rfloor$, $m=\lfloor y\rfloor$ and $u,v\in[0,1)$. Then
\[
\floor{x+y}=m+n+\floor{u+v}
\]
so the left-hand side equals
\[
m + n + m+n+\floor{u+v} = 2m + 2n + \floor{u+v}
\]
Also,
\[
\floor{2x}+\floor{2y}=\floor{2n+2u}+ \floor{2m+2v}= 2m + 2n + \floor{2u}+\floor{2v}
\]
Thus the inequality reduces to
\[
\floor{u+v}\leq  \floor{2u}+\floor{2v}
\]
for $u,v\in[0,1)$. If $u+v<1$ the left side is $0$ and the right side is $\ge0$, so the inequality holds. If $u+v\ge1$ then at least one of $u,v$ is $\ge\tfrac12$, hence at least one of $\lfloor2u\rfloor,\lfloor2v\rfloor$ equals $1$, so the right side is $\ge1$ while the left side equals $1$. In all cases the inequality holds, proving the claim.
\end{enumerate}
}

\item (Rosen 2.3-79)
\begin{enumerate}
\item[a)] \streasy Show that if a set $S$ has cardinality $m$, where $m$ is a positive integer, then there is a one-to-one correspondence between $S$ and the set $\{1,2,\ldots,m\}$.
\solution{
By definition, to say that $S$ has cardinality $m$ is to say that $S$ has exactly $m$ distinct elements. Therefore we can  enumerate: assign the first object to 1, the second to 2, and so on. This provides the one-to-one correspondence.
}
\item[b)] \strmedium Show that if $S$ and $T$ are two sets each with $m$ elements, where $m$ is a positive integer, then there is a one-to-one correspondence between $S$ and $T$.
\end{enumerate}
\solution{By part (a), there is a bijection f from $S$ to $\{1,2, \ldots ,m\}$ and a bijection $g$ from $T$ to $\{1,2, \ldots ,m\}$. Hence $g^{-1}$ is a bijection from $\{1,2, \ldots ,m\}$ to $T$. Then the composition $g^{-1}\circ f$ is the desired bijection from $S$ to $T$.
}

\item (Rosen 2.4-3) What are the terms $a_{0}, a_{1}, a_{2}$ and $a_{3}$ 
of the sequence $\{a_{n}\}$ where $a_{n}$ is given by

\begin{multicols}{2}
\begin{enumerate}
\item \streasy $2^{n}+1$

\item \streasy $(n+1)^{n+1}$

\item \streasy $\floor{\frac{n}{2}}$

\item \streasy $\floor{\frac{n}{2}} + \ceil{\frac{n}{2}}$
\end{enumerate}
\end{multicols}

\solution{
\begin{enumerate}
\item $a_{0} = 2$, $a_{1} = 3$, $a_{2} = 5$, $a_{3} = 9$

\item $a_{0} = 0$, $a_{1} = 4$, $a_{2} = 27$, $a_{3} = 256$

\item $a_{0} = 0$, $a_{1} = 0$, $a_{2} = 1$, $a_{3} = 1$

\item $a_{0} = 0$, $a_{1} = 1$, $a_{2} = 1$, $a_{3} = 3$
\end{enumerate}
}

\item (Rosen 2.4-5) List the first 10 terms of these
sequences.

\begin{enumerate}
\item \streasy the sequence starting with 2 where each term
is 3 more than the previous one;
\solution{
$2,5,8,11,14,17,20,23,26,29,\ldots$
}

\item \streasy the sequence listing each positive integer
three times in increasing order;
\solution{
$1,1,1,2,2,2,3,3,3,4,\ldots$
}

\item \streasy the sequence listing all positive odd integers
in increasing order, listing each odd twice;
\solution{
$1,1,3,3,5,5,7,7,9,9,\ldots$
}

\item \streasy the sequence whose $n$-th term is $n!-2^{n}$;
\solution{
$-1,-2,-2,8,88,656,4912,40064,362368,3627776,\ldots$
}

\item \streasy the sequence starting with 3 where each
subsequent term is twice the previous term;
\solution{
$3,6,12,24,48,96,192,384,768,1\,536,3\,072,6\,144,\ldots$
}

\item \streasy the sequence whose first term is 2, the second is 4,
and each next term is the sum of the previous two terms;
\solution{
$2,4,6,10,16,26,42,68,110,178,\ldots$
}

\item \streasy the sequence whose $n$-th term is the number
of bits in the binary representation of $n$;
\solution{
$1,2,2,3,3,3,3,4,4,4,\ldots$
}

\item \streasy the sequence whose $n$-th term is the number
of letters in the Portuguese word for $n$.
\solution{
$2,4,4,6,5,4,4,4,4,3,\ldots$
}
\end{enumerate}

\item (Rosen 2.4-26) For each of the integer lists below, 
give a simple formula or rule that generates the terms of a sequence of 
integers that starts with the given list. 
Assuming your formula is correct, give the next three elements 
of the sequence.
\begin{enumerate}
\item \streasy $3,6,11,18,27,38,51,66,83,102,...$
\solution{
The sequence starts with 3 and each next term is the previous one plus the next odd number, starting with 3.

Closed formula for $i \geq 1$:
\begin{align*}
a_{i} 
&= 2 + \sum_{k=1}^{i}(2i-1) \\
&= 2 + 2\sum_{k=1}^{i} i - \sum_{k=1}^{i} 1 \\
&= 2 + 2 \frac{(i+1)i}{2} - i \\
&= 2 + i^{2}
\end{align*}

Recursive formula:
$$
\begin{cases}
a_{1} = 3, & \\
a_{i} = a_{i-1} + (2i-1), & \text{for $i \geq 2$}
\end{cases}
$$
}

\item \streasy $7,11,15,19,23,27,31,35,39,43,...$
\solution{
Closed formula for $i \geq 1$:
\begin{align*}
a_{i}
&= 7 + 4(i-1) 
\end{align*}

Recursive formula:
$$
\begin{cases}
a_{1} = 7, & \\
a_{i} = a_{i-1} + 4, & \text{for $i \geq 2$}
\end{cases}
$$
}

\item \streasy $1,10,11,100,101,110,111,1000,1001,1010,1011,...$
\solution{
The positive integers in order, containing only the digits $0$ and $1$.
}

\item \streasy $1,2,2,2,3,3,3,3,3,5,5,5,5,5,5,5,...$
\solution{
The non-decreasing sequence of positive primes, where the $i$-th prime appears $2i-1$ times.
}

\item \strhard $0,2,8,26,80,242,728,2\,186,6\,560,19\,682,...$
\solution{
The sequence where the first term is $0$, the second is $2$, and each term is the previous term plus 3 times the difference between the two previous terms.

Recursive formula:
$$
\begin{cases}
a_{1} = 0, & \\
a_{2} = 2, & \\
a_{i} = a_{i-1} + 3(a_{i-1} - a_{i-2}), & \text{for $i \geq 3$}
\end{cases}
$$
}

\item \strhard $1,3,15,105,945,10\,395,135\,135,2\,027\,025,34\,459\,425,...$
\solution{
The sequence where the first term is $1$ and the $n$-th term is the previous term multiplied by the $(n+1)$-th positive odd integer.
}

\item \strmedium $1,0,0,1,1,1,0,0,0,0,1,1,1,1,1,...$
\solution{
One $1$, followed by two $0$s, followed by three $1$s, followed by four $0$s, and so on...
}
\end{enumerate} 

\item (Rosen 2.4-43) What are the values of the following products:
\begin{enumerate}
\item \streasy $\prod_{i=0}^{13}i$
\solution{
$\prod_{i=0}^{13} i = 0 \cdot 1 \cdot \ldots \cdot 13 = 0$
}

\item \streasy $\prod_{i=5}^{10}i$
\solution{
$\prod_{i=5}^{10} i = 5 \cdot 6 \cdot 7 \cdot 8 \cdot 9 \cdot 10 = 151{,}200$
}

\item \strmedium $\prod_{i=0}^{99}(-1)^{i}$
\solution{
$\prod_{i=0}^{99}(-1)^{i} = (-1) \cdot 1 \cdot (-1) \cdot \ldots \cdot (-1) = (-1)^{50} \cdot 1^{50} = 1 \cdot 1 = 1$
}

\item \strmedium $\prod_{i=0}^{11}2$
\solution{
$\prod_{i=0}^{11} 2 = 2 \cdot 2 \cdot \ldots \cdot 2 = 2^{12}$
}
\end{enumerate}

\item (Rosen 2.5-1) Determine if each of the sets below is 
finite, countably infinite, or uncountable.
For countably infinite sets, list the first 10 elements in an enumeration.

\begin{enumerate}
\item \streasy the negative integers
\solution{
Countably infinite: $-1, -2, -3, -4, -5, -6, -7, -8, -9, -10, \ldots$
}

\item \streasy the even integers
\solution{
Countably infinite: $0, 2, -2, 4, -4, 6, -6, 8, -8, 10, \ldots$
}

\item \streasy the integers less than 100
\solution{
Countably infinite: $99, 98, 97, 96, 95, 94, 93, 92, 91, 90, \ldots$
}

\item \streasy the real numbers between $0$ and $\nicefrac{1}{2}$
\solution{
Uncountable.
}

\item \streasy the positive integers less than $1\,000\,000\,000$
\solution{
Finite.
}

\item \streasy the integers that are multiples of 7
\solution{
Countably infinite: $0, 7, -7, 14, -14, 21, -21, 28, -28, 35, \ldots$
}
\end{enumerate}

\item (Rosen 2.5-2) Determine if each of the sets below is countable or uncountable. 
For countable sets, list the first 10 elements in an enumeration.
\begin{enumerate}
\item \streasy the integers greater than 10;
\solution{
Countable: $11, 12, 13, 14, 15, 16, 17, 18, 19, 20, \ldots$
}

\item \streasy the negative odd integers;
\solution{
Countable: $-1, -3, -5, -7, -9, -11, -13, -15, -17, -19, \ldots$
}

\item \streasy the real numbers between 0 and 2;
\solution{
Uncountable: the uncountable interval $(0,1)$ is contained in this set.
}

\item \streasy the integers that are multiples of 10.
\solution{
Countable: $0, 10, -10, 20, -20, 30, -30, 40, -40, 50, \ldots$
}
\end{enumerate}

\item (Rosen 2.5-11) Give an example of two uncountable sets $A$ and $B$ such that $A \cap B$ is

\begin{enumerate}
\item \strmedium finite

\item \strmedium countably infinite

\item \strmedium uncountable
\end{enumerate}

\solution{
\begin{enumerate}
\item $A = [1,2]$, $B = [3,4]$, $A \cap B = \varnothing$ is finite.

\item $A = [1,2] \cup \mathbb{Z}^{+}$, $B = [3,4] \cup \mathbb{Z}^{+}$, 
$A \cap B = \mathbb{Z}^{+}$ is countably infinite.

\item $A = [1,3]$, $B = [2,4]$, $A \cap B = [2,3]$ is uncountable.
\end{enumerate}
}

\item (Rosen 2.5-34) Determine if each of the sets below is countable or not. 
For countable sets, list the first 10 elements in an enumeration.
\begin{enumerate}
\item \streasy the integers not divisible by 3.
\solution{
Countable: $1, 2, 4, 5, 7, 8, 10, 11, 13, 14, \ldots$ \\
or $1, -1, 2, -2, 4, -4, 5, -5, 7, -7, \ldots$ 
}

\item \streasy the integers divisible by 5 but not by 7.
\solution{
Countable: $0, 5, 10, 15, 20, 25, 30, 35, 40, 45, \ldots$ \\
or $0, 5, -5, 10, -10, 15, -15, 20, -20, 25, \ldots$
}

\item \strmedium the real numbers with decimal representation containing only 1s.
\solution{
Countable: $1$, $.1$, $11$, $1.1$, $.11$, $111$, $11.1$, $1.11$, $.111$, $1111$, $\ldots$ \\
or $1$, $-1$, $.1$, $-.1$, $11$, $-11$, $1.1$, $-1.1$, $.11$, $-.11$, $\ldots$
}

\item \strmedium the real numbers with decimal representation containing only 1s or 9s.
\solution{
Uncountable: one can construct a bijection between the real numbers with decimal representation containing only 1s or 9s and the real numbers in binary representation. Therefore, the two sets have the same cardinality; that is, they are uncountable.
}
\end{enumerate}

\item (Rosen 2.5-28) \strhard Show that the set $\mathbb{Z}^{+} \times \mathbb{Z}^{+}$ is countable.

\solution{
We can enumerate the elements of $\mathbb{Z}^{+} \times \mathbb{Z}^{+}$
as in the table below.
Each row of the table corresponds to an element of $\mathbb{Z}^{+}$, and 
each column also corresponds to an element of $\mathbb{Z}^{+}$, such that 
each cell of the table corresponds to the ordered pair formed by the 
corresponding row and column.
The circles represent one possible enumeration of the elements of $\mathbb{Z}^{+} \times \mathbb{Z}^{+}$.

\begin{table}[!htb]
\renewcommand{\arraystretch}{1.5}
\centering
\begin{tabular}{|c||c|c|c|c|c|c|}
\hline
\textbf{$\mathbb{Z}^{+} \times \mathbb{Z}^{+}$} & $1$ & $2$ & $3$ & $4$ & $5$ & $\ldots$ \\
\hline \hline
1 & \orgcellA{1}{1}{1} & \orgcellA{1}{2}{2} & \orgcellA{1}{3}{4} & \orgcellA{1}{4}{7} & \orgcellA{1}{5}{11} & $\ldots$ \\ \hline
2 & \orgcellA{2}{1}{3} & \orgcellA{2}{2}{5} & \orgcellA{2}{3}{8} & \orgcellA{2}{4}{12} & \orgcellB{2}{5}{} & $\ldots$ \\ \hline
3 & \orgcellA{3}{1}{6} & \orgcellA{3}{2}{9} & \orgcellA{3}{3}{13} & \orgcellB{3}{4}{} & \orgcellB{3}{5}{} & $\ldots$ \\ \hline
4 & \orgcellA{4}{1}{10} & \orgcellA{4}{2}{14} & \orgcellB{4}{3}{} & \orgcellB{4}{4}{} & \orgcellB{4}{5}{} & $\ldots$ \\ \hline
5 & \orgcellA{5}{1}{15} & \orgcellB{5}{2}{} & \orgcellB{5}{3}{} & \orgcellB{5}{4}{} & \orgcellB{5}{5}{} & $\ldots$ \\ \hline
$\ldots$ & $\ldots$ & $\ldots$ & $\ldots$ & $\ldots$ & $\ldots$ & $\ldots$ \\ \hline
\end{tabular}
\renewcommand{\arraystretch}{1.}
\end{table}
}


\end{enumerate}



\end{document}
