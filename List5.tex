\input{hmwrk_header.tex}

%====================================================================
% Commands particular to this file
%====================================================================

%--------------------------------------------------------------------

\begin{document}

%====================================================================
\homeworktitle{Recursion and Structural Induction}{Rosen - Chapter 5}

\crossline

\paragraph{Required reading for this list:}
\emph{Discrete Mathematics and Its Applications} (Rosen, 7\textsuperscript{th} Edition):
\begin{itemize}
\item Chapter 5.3: \emph{Recursive Definitions and Structural Induction}
\item Chapter 5.4: \emph{Recursive Algorithms}
\end{itemize}

\noteondifficultylevel

\crossline
%--------------------------------------------------------------------

\begin{enumerate}

\item (Rosen 5.3-7) Give a recursive definition for the sequence
${a_n}$, $n=1,2,3,...$ if

\begin{enumerate}
\item \streasy $a_{n} = 6n$.

\item \strmedium $a_{n} = 2n+1$.

\item \strmedium $a_{n} = 10^{n}$.

\item \streasy $a_{n} = 5$.

\end{enumerate}

\item (Rosen 5.3-8) Give a recursive definition for the sequence
${a_n}$, $n=1,2,3,...$ if
\begin{enumerate}
\item \streasy $a_{n} = 4n-2$.

\item \strmedium $a_{n} = 1+(-1)^n$.

\item \strmedium $a_{n} = n(n+1)$.

\item \streasy $a_{n} = n^2$.
\end{enumerate}

\item (Rosen 5.3-25) Give a recursive definition of:

\begin{enumerate}
\item \streasy the set of even integers.


\item \streasy the set of positive integers congruent to 2 modulo 3 
(that is, the positive integers that have remainder 2 when divided by 3).

\item \strmedium the set of positive integers not divisible by 5.
\end{enumerate} 

\item (Rosen 5.3-27) Let $S$ be a subset of ordered pairs 
of integers, defined recursively by

\emph{Base step:} $(0,0)\in S$,

\emph{Recursive step:} If $(a,b)\in S$, then $(a,b+1)\in S$, 
$(a+1,b+1)\in S$ and $(a+2, b+1) \in S$.

\begin{enumerate}
\item \streasy List the elements of $S$ produced by the first four applications 
of the recursive definition.

\item \strmedium Use structural induction to show that $a \leq 2b$ whenever 
$(a,b)\in S$.
\end{enumerate}

\item (Rosen 5.3-28) Give a recursive definition for each of the 
sets of ordered pairs of positive integers. 
(Hint: Plot the points on the plane and look for lines containing 
the points of the set.)

\begin{enumerate}
\item \streasy $S=\{(a,b)  \mid  a,b\in \mathbb{Z}^+, a+b \text{~is odd}\}$

\item \strmedium $S=\{(a,b)  \mid  a,b\in \mathbb{Z}^+, a  \mid  b\}$

\item \strmedium $S=\{(a,b)  \mid  a,b\in \mathbb{Z}^+, 3  \mid  a+b\}$
\end{enumerate}

\item (Rosen 5.3-29) Give a recursive definition for each of the 
sets of ordered pairs of positive integers. 
(Hint: Plot the points on the plane and look for lines containing 
the points of the set.)

\begin{enumerate}
\item \strmedium $S=\{(a,b)  \mid  a,b\in \mathbb{Z}^+, a+b \text{~is even}\}$

\item \strmedium $S=\{(a,b)  \mid  a,b\in \mathbb{Z}^+, \text{~$a$ or $b$ is odd}\}$
\end{enumerate}

\item (Rosen 5.3-33) 
\begin{enumerate}
\item \strmedium Give a recursive definition of the function $m(s)$, which returns the smallest
digit in a string $s$ of decimal digits.
(Ex: $m(3459367)=3$, $m(12)=1$, $m(979)=7$).

\item \strhard Use induction to show that $m(st) = \min(m(s),m(t))$.
(Hint: Try induction on the length of the string $s$, assuming $s$ and $t$ are nonempty decimal-digit strings.)

\end{enumerate}

\item \strhard (Rosen 5.3-35) Give a recursive definition for the reverse of 
a string. 
(Hint: first define the reverse of the empty string $\lambda$. 
Then write a string $w$ of length $n+1$ as $xy$, where $x$ 
is a string of length $n$, and express the reverse of $w$ in terms 
of $x^R$ and $y$.)

\item \strhard (Rosen 5.3-43) Let $T$ be a complete binary tree 
(that is, a tree in which all internal nodes have exactly two 
child nodes), let $n(T)$ be the number of nodes in the tree $T$, and let $h(T)$ 
be the height (that is, the longest path from the root to a leaf of the tree) 
of $T$.

Use structural induction to show that $n(T) \geq 2 h(T)+1$.

\item  (Rosen 5.3-48 to 55) Consider the following inductive definition of a version of Ackermann's function:
\[
A(m,n)=
\left\{
\begin{array}{ll}
2n & \mbox{ if } m=0\\
0 & \mbox{ if } m\geq 1 \mbox{ and } n=0\\
2 & \mbox{ if } m\geq 1 \mbox{ and } n=1\\
A(m-1,A(m,n-1)) & \mbox{ if } m\geq 1 \mbox{ and } n\geq 2\\
\end{array}
\right.
\]
\begin{enumerate}
\item[4.3-48] \streasy Find the following values of Ackermann's function: $A(1,0),A(0,1),A(1,1),A(2,2)$.
\item[4.3-49] \streasy Show that $A(m, 2) = 4$ whenever $m \geq 1$.

\item[4.3-50] \streasy Show that $A(1, n) = 2^n$ whenever $n \geq 1$.

\item[4.3-53] \strhard Prove that $A(m,n+1) > A(m,n)$ for all $m,n\ge 0$. 

\item[4.3-54]  \strmedium Prove that $A(m + 1, n) \geq A(m, n)$ whenever $m$ and $n$ are nonnegative integers.

\item[4.3-55] \strmedium Prove that $A(i, j ) \geq j$ whenever $i$ and $j$ are nonnegative
integers.
\end{enumerate}

\item \streasy (Rosen 5.4-7)  Give a recursive algorithm for computing $n.x$ whenever $n$ is a positive integer and $n$ is an integer, using just addition.

\item \streasy (Rosen 5.4-10) Give a recursive algorithm for finding the maximum of a finite set of integers, making use of the fact that the maximum of $n$ integers is the larger of the last integer in the list and the maximum of the first $n - 1$ integers in the list.

\item \streasy (Rosen 5.4-12)  Devise a recursive algorithm for finding $x^n \pmod{m}$ whenever $n, x$, and $m$ are positive integers based on the fact that $x^n \pmod{m}= (x^{n-1} \pmod{m}\cdot x\pmod{m})\pmod{m}$.

\item (Rosen 5.4-51 and 55) The quick sort is an efficient algorithm. To sort $a_1,a_2,\ldots,a_n$, this algorithm begins by taking the first element $a_1$ and forming two sublists, the first containing those elements that are less than $a_1$, in the order they arise, and the second containing those elements greater than $a_1$, in the order they arise. Then $a_1$ is put at the end of the first sublist. This procedure is repeated recursively for each sublist, until all sublists contain one item. The ordered list of $n$ items is obtained by combining the sublists of one item in the order they occur.
\begin{itemize}
\item[5.4-51] \strmedium  Let $a_1,a_2,\ldots,a_n$ be a list of n distinct real numbers. How many comparisons are needed to form two sublists from this list, the first containing elements less than $a_1$ and the second containing elements greater than $a_1$?
\item[5.4-55] \strhard Determine the worst-case complexity of the quick sort algorithm in terms of the number of comparisons used.

\end{itemize}

\end{enumerate}

\end{document}

%\item A non-recursive formula for the Fibonacci sequence is
%\begin{equation*}
%f_{n} = \frac{1}{\sqrt{5}} ( \frac{1+\sqrt{5}}{2} )^{n} 
%	    - \frac{1}{\sqrt{5}} ( \frac{1-\sqrt{5}}{2} )^{n}.
%\end{equation*}
%
%The numerical value of $\frac{1+\sqrt{5}}{2}$ is approximately $1.61803398$,
%while its reciprocal, $1/(\frac{1+\sqrt{5}}{2})$, is approximately 
%$0.61803398$.
%Consider the sequence $g_n = \frac{1}{\sqrt{5}} ( \frac{1+\sqrt{5}}{2} )^n $.
%This sequence never results in an integer, but according to the previous item 
%it should be a good approximation of $f_{n}$.
%
%The numerical values (to 4 decimal places) are:
%\begin{center}
%\begin{tabular}{c|c|c}
%$n$ & $g_n$ & $f_n$\\ \hline
%0 & 0.4472 & 0 \\
%1 & 0.7236 & 1 \\
%2 & 1.1708 & 1 \\
%3 & 1.8944 & 2 \\
%4 & 3.0652 & 3 \\
%5 & 4.9597 & 5 \\
%6 & 8.0249 & 8  \\
%7 & 12.9846 & 13 \\
%8 & 21.0095 & 21 \\
%\end{tabular}
%\end{center}
%
%Prove that $f_n = [ \frac{1}{\sqrt{5}} ( \frac{1+\sqrt{5}}{2} )^n ] $,
%where $[x]$ is the function that maps a real number $x$ to the nearest integer
%(in other words, for $x \in \mathbb{R}$ the value $[x] \in \mathbb{Z}$, and $ [x] - \nicefrac{1}{2} \le x < [x]+ \nicefrac{1}{2}$).
%Hint: consider $ \varepsilon_n = - \frac{1}{\sqrt{5}} ( \frac{1-\sqrt{5}}{2} )^n $.
%
%\solution{
%}

%\item (Rosen 4.3-25) Give a recursive definition for the sequence
%${a_n}$, $n=0,1,2,...$ if
%\begin{enumerate}
%\item \streasy $a_{n} = 3n+1$.
%\solution{
%First, let's determine $a_{0}$.
%\[\begin{array}
%a_{0}
%&= 3(0) + 1 \
%&= 1
%\end{array}\]
%Now let's determine $a_{n}$ for $n \geq 1$.
%\[\begin{array}
%a_{n-1}
%&= 3(n-1) + 1 \
%&= 3n - 3 + 1 \
%&= 3n - 2 \
%&= (3n + 1) - 3 \
%&= a_{n} - 3
%\end{array}\]
%Therefore
%\[\begin{array}
%a_{n}
%&= a_{n-1} + 3,
%\end{array}\]
%and the final answer is
%\[\begin{array}
%\[\begin{cases}
%a_{0} = 1, \
%a_{n} = a_{n-1} + 3, & \text{for $n\geq1$.}
%\end{cases}\]
%\end{array}\]
%}
%
%\item \strmedium $a_{n} = 2^{n+1}$.
%\solution{
%First, let's determine $a_{0}$.
%\[\begin{array}
%a_{0}
%&= 2^{0+1} \
%&= 2
%\end{array}\]
%Now let's determine $a_{n}$ for $n \geq 1$.
%\[\begin{array}
%a_{n-1}
%&= 2^{(n-1)+1} \
%&= 2^{n} \
%&= \frac{2^{n+1}}{2} \
%&= \frac{a_{n}}{2}
%\end{array}\]
%Therefore
%\[\begin{array}
%a_{n}
%&= 2 a_{n-1},
%\end{array}\]
%and the final answer is
%\[\begin{array}
%\[\begin{cases}
%a_{0} = 2, \
%a_{n} = 2 a_{n-1}, & \text{for $n\geq1$.}
%\end{cases}\]
%\end{array}\]
%}
%
%\item \strmedium $a_{n} = 5(-2)^n$.
%\solution{
%First, let's determine $a_{0}$.
%\[\begin{array}
%a_{0}
%&= 5(-2)^0 \
%&= 5
%\end{array}\]
%Now let's determine $a_{n}$ for $n \geq 1$.
%\[\begin{array}
%a_{n-1}
%&= 5(-2)^{n-1} \
%&= 5\frac{(-2)^n}{-2} \
%&= -\frac{a_{n}}{2}
%\end{array}\]
%Therefore
%\[\begin{array}
%a_{n}
%&= -2 a_{n-1},
%\end{array}\]
%and the final answer is
%\[\begin{array}
%\[\begin{cases}
%a_{0} = 5, \
%a_{n} = -2 a_{n-1}, & \text{for $n\geq1$.}
%\end{cases}\]
%\end{array}\]
%}
%\end{enumerate}
%
%\item (Rosen 4.3-26) Give a recursive definition for the sequence
%${a_n}$, $n=0,1,2,...$ if
%\begin{enumerate}
%\item \streasy $a_{n} = n^2 + 1$.
%\solution{
%First, let's determine $a_{0}$.
%\[\begin{array}
%a_{0}
%&= (0)^2 + 1 \
%&= 1
%\end{array}\]
%Now let's determine $a_{n}$ for $n \geq 1$.
%\[\begin{array}
%a_{n-1}
%&= (n-1)^2 + 1 \
%&= n^2 - 2n + 1 + 1 \
%&= n^2 - 2n + 2 \
%&= (n^2 + 1) - 2n + 1 \
%&= a_{n} - 2n + 1
%\end{array}\]
%Therefore
%\[\begin{array}
%a_{n}
%&= a_{n-1} + 2n - 1,
%\end{array}\]
%and the final answer is
%\[\begin{array}
%\[\begin{cases}
%a_{0} = 1, \
%a_{n} = a_{n-1} + 2n - 1, & \text{for $n\geq1$.}
%\end{cases}\]
%\end{array}\]
%}
%
%\item \strmedium $a_{n} = n! + 1$.
%\solution{
%First, let's determine $a_{0}$.
%\[\begin{array}
%a_{0}
%&= 0! + 1 \
%&= 1 + 1 \
%&= 2
%\end{array}\]
%Now let's determine $a_{n}$ for $n \geq 1$.
%\[\begin{array}
%a_{n-1}
%&= (n-1)! + 1 \
%&= \frac{n!}{n} + 1 \
%&= \frac{1}{n}(a_{n} - 1) + 1 \
%&= \frac{a_{n} - 1 + n}{n} \
%&= \frac{a_{n} + n - 1}{n}
%\end{array}\]
%Therefore
%\[\begin{array}
%a_{n}
%&= n a_{n-1} - n + 1,
%\end{array}\]
%and the final answer is
%\[\begin{array}
%\[\begin{cases}
%a_{0} = 2, \
%a_{n} = n a_{n-1} - n + 1, & \text{for $n\geq1$.}
%\end{cases}\]
%\end{array}\]
%}
%\end{enumerate}
%
%\item (Rosen 4.3-27) Give a recursive definition for the sequence
%${a_n}$, $n=0,1,2,...$ if
%\begin{enumerate}
%\item \streasy $a_{n} = 2^n + 3^n$.
%\solution{
%First, let's determine $a_{0}$.
%\[\begin{array}
%a_{0}
%&= 2^0 + 3^0 \
%&= 1 + 1 \
%&= 2
%\end{array}\]
%Now let's determine $a_{n}$ for $n \geq 1$.
%\[\begin{array}
%a_{n-1}
%&= 2^{n-1} + 3^{n-1} \
%&= \frac{2^n}{2} + \frac{3^n}{3} \
%&= \frac{1}{6}(3\cdot2^n + 2\cdot3^n) \
%&= \frac{1}{6}[3(a_{n} - 3^n) + 2\cdot3^n] \
%&= \frac{1}{6}[3a_{n} - 3\cdot3^n + 2\cdot3^n] \
%&= \frac{1}{6}[3a_{n} - 3^n]
%\end{array}\]
%Therefore
%\[\begin{array}
%a_{n}
%&= 2 a_{n-1} + 3^{n-1},
%\end{array}\]
%and the final answer is
%\[\begin{array}
%\[\begin{cases}
%a_{0} = 2, \
%a_{n} = 2 a_{n-1} + 3^{n-1}, & \text{for $n\geq1$.}
%\end{cases}\]
%\end{array}\]
%}
%
%\item \strmedium $a_{n} = 2^{n+1} - 3^{n}$.
%\solution{
%First, let's determine $a_{0}$.
%\[\begin{array}
%a_{0}
%&= 2^{0+1} - 3^0 \
%&= 2 - 1 \
%&= 1
%\end{array}\]
%Now let's determine $a_{n}$ for $n \geq 1$.
%\[\begin{array}
%a_{n-1}
%&= 2^{(n-1)+1} - 3^{n-1} \
%&= 2^n - 3^{n-1} \
%&= \frac{1}{3}(3\cdot2^n - 3^n) \
%&= \frac{1}{3}[3(2^n - 3^n) + 2\cdot3^n] \
%&= \frac{1}{3}[3 a_{n} + 2\cdot3^n]
%\end{array}\]
%Therefore
%\[\begin{array}
%a_{n}
%&= \frac{3 a_{n-1} - 2\cdot3^{n-1}}{3},
%\end{array}\]
%and the final answer is
%\[\begin{array}
%\[\begin{cases}
%a_{0} = 1, \
%a_{n} = \frac{3 a_{n-1} - 2\cdot3^{n-1}}{3}, & \text{for $n\geq1$.}
%\end{cases}\]
%\end{array}\]
%}
%\end{enumerate}
%
%\item (Rosen 4.3-28) Give a recursive definition for the sequence
%${a_n}$, $n=0,1,2,...$ if
%\begin{enumerate}
%\item \streasy $a_{n} = 4n^2 + 2$.
%\solution{
%First, let's determine $a_{0}$.
%\[\begin{array}
%a_{0}
%&= 4(0)^2 + 2 \
%&= 2
%\end{array}\]
%Now let's determine $a_{n}$ for $n \geq 1$.
%\[\begin{array}
%a_{n-1}
%&= 4(n-1)^2 + 2 \
%&= 4(n^2 - 2n + 1) + 2 \
%&= 4n^2 - 8n + 4 + 2 \
%&= 4n^2 - 8n + 6 \
%&= (4n^2 + 2) - 8n + 4 \
%&= a_{n} - 8n + 4
%\end{array}\]
%Therefore
%\[\begin{array}
%a_{n}
%&= a_{n-1} + 8n - 4,
%\end{array}\]
%and the final answer is
%\[\begin{array}
%\[\begin{cases}
%a_{0} = 2, \
%a_{n} = a_{n-1} + 8n - 4, & \text{for $n\geq1$.}
%\end{cases}\]
%\end{array}\]
%}
%
%\item \strmedium $a_{n} = 7n^3$.
%\solution{
%First, let's determine $a_{0}$.
%\[\begin{array}
%a_{0}
%&= 7(0)^3 \
%&= 0
%\end{array}\]
%Now let's determine $a_{n}$ for $n \geq 1$.
%\[\begin{array}
%a_{n-1}
%&= 7(n-1)^3 \
%&= 7(n^3 - 3n^2 + 3n - 1) \
%&= 7n^3 - 21n^2 + 21n - 7 \
%&= (7n^3) - (21n^2 - 21n + 7) \
%&= a_{n} - (21n^2 - 21n + 7)
%\end{array}\]
%Therefore
%\[\begin{array}
%a_{n}
%&= a_{n-1} + 21n^2 - 21n + 7,
%\end{array}\]
%and the final answer is
%\[\begin{array}
%\[\begin{cases}
%a_{0} = 0, \
%a_{n} = a_{n-1} + 21n^2 - 21n + 7, & \text{for $n\geq1$.}
%\end{cases}\]
%\end{array}\]
%}
%\end{enumerate}
%
%\item (Rosen 4.3-29) Give a recursive definition for the sequence
%${a_n}$, $n=0,1,2,...$ if
%\begin{enumerate}
%\item \streasy $a_{n} = 2n + (-1)^n$.
%\solution{
%First, let's determine $a_{0}$.
%\[\begin{array}
%a_{0}
%&= 2(0) + (-1)^0 \
%&= 1
%\end{array}\]
%Now let's determine $a_{n}$ for $n \geq 1$.
%\[\begin{array}
%a_{n-1}
%&= 2(n-1) + (-1)^{n-1} \
%&= 2n - 2 + (-1)(-1)^n \
%&= 2n - 2 - (-1)^n \
%&= (2n + (-1)^n) - 2 - 2(-1)^n \
%&= a_{n} - 2 - 2(-1)^n
%\end{array}\]
%Therefore
%\[\begin{array}
%a_{n}
%&= a_{n-1} + 2 + 2(-1)^n,
%\end{array}\]
%and the final answer is
%\[\begin{array}
%\[\begin{cases}
%a_{0} = 1, \
%a_{n} = a_{n-1} + 2 + 2(-1)^n, & \text{for $n\geq1$.}
%\end{cases}\]
%\end{array}\]
%}
%
%\item \strmedium $a_{n} = 3^n + n$.
%\solution{
%First, let's determine $a_{0}$.
%\[\begin{array}
%a_{0}
%&= 3^0 + 0 \
%&= 1
%\end{array}\]
%Now let's determine $a_{n}$ for $n \geq 1$.
%\[\begin{array}
%a_{n-1}
%&= 3^{n-1} + (n-1) \
%&= \frac{3^n}{3} + n - 1 \
%&= \frac{a_{n} - n}{3} + n - 1 \
%&= \frac{a_{n} - n + 3n - 3}{3} \
%&= \frac{a_{n} + 2n - 3}{3}
%\end{array}\]
%Therefore
%\[\begin{array}
%a_{n}
%&= 3a_{n-1} - 2n + 3,
%\end{array}\]
%and the final answer is
%\[\begin{array}
%\[\begin{cases}
%a_{0} = 1, \
%a_{n} = 3a_{n-1} - 2n + 3, & \text{for $n\geq1$.}
%\end{cases}\]
%\end{array}\]
%}
%\end{enumerate}
%\item (Rosen 4.3-30) Give a recursive definition for the sequence
%${a_n}$, $n=0,1,2,...$ if
%
%\begin{enumerate}
%\item \streasy $a_{n} = n! + 1$.
%\solution{
%First, let's determine $a_{0}$.
%\[\begin{array}
%a_{0}
%&= 0! + 1 \
%&= 2
%\end{array}\]
%Now let's determine $a_{n}$ for $n \geq 1$.
%\[\begin{array}
%a_{n-1}
%&= (n-1)! + 1 \
%&= \frac{n!}{n} + 1 \
%&= \frac{a_{n} - 1}{n} + 1
%\end{array}\]
%Therefore
%\[\begin{array}
%a_{n}
%&= n(a_{n-1} - 1) + 1,
%\end{array}\]
%and the final answer is
%\[\begin{array}
%\[\begin{cases}
%a_{0} = 2, \
%a_{n} = n(a_{n-1} - 1) + 1, & \text{for $n\geq1$.}
%\end{cases}\]
%\end{array}\]
%}
%
%\item \strmedium $a_{n} = 2^{n} + (-1)^{n}$.
%\solution{
%First, let's determine $a_{0}$.
%\[\begin{array}
%a_{0}
%&= 2^{0} + (-1)^{0} \
%&= 2
%\end{array}\]
%Now let's determine $a_{n}$ for $n \geq 1$.
%\[\begin{array}
%a_{n-1}
%&= 2^{n-1} + (-1)^{n-1} \
%&= \frac{2^n}{2} - (-1)^{n} \
%&= \frac{a_{n} - (-1)^{n}}{2} - (-1)^{n}
%\end{array}\]
%Therefore
%\[\begin{array}
%a_{n}
%&= 2a_{n-1} + (-1)^{n+1},
%\end{array}\]
%and the final answer is
%\[\begin{array}
%\[\begin{cases}
%a_{0} = 2, \
%a_{n} = 2a_{n-1} + (-1)^{n+1}, & \text{for $n\geq1$.}
%\end{cases}\]
%\end{array}\]
%}
%\end{enumerate}
%
%\item (Rosen 4.3-31) Give a recursive definition for the sequence
%${a_n}$, $n=0,1,2,...$ if
%
%\begin{enumerate}
%\item \streasy $a_{n} = n^{2} + n + 1$.
%\solution{
%First, let's determine $a_{0}$.
%\[\begin{array}
%a_{0}
%&= 0^{2} + 0 + 1 \
%&= 1
%\end{array}\]
%Now let's determine $a_{n}$ for $n \geq 1$.
%\[\begin{array}
%a_{n-1}
%&= (n-1)^{2} + (n-1) + 1 \
%&= n^{2} - 2n + 1 + n - 1 + 1 \
%&= n^{2} - n + 1 \
%&= (n^{2} + n + 1) - 2n \
%&= a_{n} - 2n
%\end{array}\]
%Therefore
%\[\begin{array}
%a_{n}
%&= a_{n-1} + 2n,
%\end{array}\]
%and the final answer is
%\[\begin{array}
%\[\begin{cases}
%a_{0} = 1, \
%a_{n} = a_{n-1} + 2n, & \text{for $n\geq1$.}
%\end{cases}\]
%\end{array}\]
%}
%
%\item \strmedium $a_{n} = 2n^{2} + 3n + 1$.
%\solution{
%First, let's determine $a_{0}$.
%\[\begin{array}
%a_{0}
%&= 2(0)^{2} + 3(0) + 1 \
%&= 1
%\end{array}\]
%Now let's determine $a_{n}$ for $n \geq 1$.
%\[\begin{array}
%a_{n-1}
%&= 2(n-1)^{2} + 3(n-1) + 1 \
%&= 2(n^{2} - 2n + 1) + 3n - 3 + 1 \
%&= 2n^{2} - 4n + 2 + 3n - 2 \
%&= 2n^{2} - n \
%&= (2n^{2} + 3n + 1) - 4n - 1 \
%&= a_{n} - 4n - 1
%\end{array}\]
%Therefore
%\[\begin{array}
%a_{n}
%&= a_{n-1} + 4n + 1,
%\end{array}\]
%and the final answer is
%\[\begin{array}
%\[\begin{cases}
%a_{0} = 1, \
%a_{n} = a_{n-1} + 4n + 1, & \text{for $n\geq1$.}
%\end{cases}\]
%\end{array}\]
%}
%\end{enumerate}
%
%\end{enumerate}
%
%\end{document}
%
