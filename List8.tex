\documentclass[11pt,a4paper]{article}

%====================================================================
% LaTeX Packages
%====================================================================
\usepackage{geometry}
\usepackage{enumerate}
\usepackage{mdwlist} % Allows for suspeding and resuming lists.
\geometry{a4paper,left=1.5cm,right=1.5cm,top=2.5cm,bottom=2cm}
\usepackage{epsf}
\usepackage{amsmath,amssymb}
\usepackage{array} % Allows fixing row width.
\usepackage{amsthm} % Allows for \qed symbol
\usepackage{graphicx}
\usepackage{nicefrac}
\usepackage{multicol}
\usepackage{xspace} % Introduces spaces after macros only when needed.
%--------------------------------------------------------------------


% Information about the course.
\newcommand{\strcourse}{Discrete Mathematics - COMP0147}
\newcommand{\strinstitution}{\includegraphics[height=1cm]{UCL-logo.jpeg}}
\newcommand{\strsemester}{2025 - Term 1}

% Information about the instructor.
\newcommand{\strinstructor}{Prof. Elaine Pimentel}
\newcommand{\stremail}{e.pimentel@ucl.ac.uk}

% Information about assigment
\newcommand{\strtypeofassignment}{List of exercises}
\newcommand{\strsolution}{(Solution)}
\newcommand{\strinstructorsolution}{Solution}
\newcommand{\streasy}{\level{Easy}}
\newcommand{\strmedium}{\level{Medium}}
\newcommand{\strhard}{\level{Hard}}
\newcommand{\strnoteondifficultylevel}{
\paragraph{Note:} 
The exercises are classified into difficulty levels:
easy, medium, and hard.
This classification, however, is only indicative.
Different people may disagree about the difficulty level of the same exercise.
Do not be discouraged if you see a difficult exercise--you may find that it is actually easy, by discovering a simpler way to solve it!
}
%--------------------------------------------------------------------

\newcommand{\level}[1]{\textbf{\texttt{[#1]}}\xspace}
\newcommand{\qm}[1]{``#1''}
\newcommand{\prop}[1]{\emph{\qm{#1}}} % Proposition in natural language
\newcommand{\lazyfrac}[2]{#1/#2}
\newcommand{\crossline}{\noindent\makebox[\linewidth]{\rule{\textwidth}{1pt}}}
\newcommand{\ie}{{\em i.e.}}
\newcommand{\eg}{{\em e.g.}}

\newcommand{\imp}{\rightarrow} % Implication symbol
\newcommand{\doubleimp}{\leftrightarrow} % Double implication symbol

\newcommand{\floor}[1]{\left\lfloor #1 \right\rfloor}
\newcommand{\ceil}[1]{\left\lceil #1 \right\rceil}

\newcommand{\green}[1]{{\color{green} #1}}
\newcommand{\red}[1]{{\color{red} #1}}


\usepackage{pifont}
\newcommand{\cmark}{\ding{51}}
\newcommand{\xmark}{\ding{55}}

\newcommand{\yes}{\green{\cmark}}
\newcommand{\no}{\red{\xmark}}


\usepackage{tikz}
\newcommand*\circled[1]{\tikz[baseline=(char.base)]{
            \node[shape=circle,draw,inner sep=2pt] (char) {#1};}}
\newcommand{\orgcellA}[3]{\begin{Large}$({#1},{#2})$\end{Large} {\circled{#3}}}
\newcommand{\orgcellB}[2]{\begin{Large}$({#1},{#2})$\end{Large} {$\ldots$}}

% We need this in order to be able to print both a handouts version
% and a solution version of the homework.
\ifdefined\hidesolution
	\newcommand{\solution}[1]{}  
\else
	\newcommand{\solution}[1]{\paragraph{\strinstructorsolution:} #1 \vspace{4mm}}
\fi

\newcommand{\homeworktitle}[2]{
\begin{center}
\begin{flushleft}
\noindent \textbf{\strinstitution \hfill \strsemester} \\
\textbf{\strcourse \hfill \strinstructor}
\end{flushleft} 
\ \\
\textbf{\MakeUppercase{\strtypeofassignment}}\\
\textsc{#1\\
(#2)}
\end{center}
}

\newcommand{\noteondifficultylevel}{
\strnoteondifficultylevel
}
%--------------------------------------------------------------------







%====================================================================
% Commands particular to this file
%====================================================================
\usepackage{graphicx}
\usepackage{multirow}
%\usepackage{bbding} % Allows for Checkmark symbol (\Checkmark with capital is bigger!) and for \XSolidBrush.
%\usepackage[usenames,dvipsnames]{color} % Allows for colors.
%\newcommand{\yes}{{\color{Green}{\Checkmark}}} % Symbol for yes. \Checkmark with capital is bigger!
%\newcommand{\no}{{\color{Red}{\XSolidBrush}}} % Symbol for no. %--------------------------------------------------------------------



\begin{document}

%====================================================================
\homeworktitle{Relations}{Rosen - Chapter 9}

\crossline

\paragraph{Required reading for this list:}
\emph{Discrete Mathematics and Its Applications} (Rosen, 7\textsuperscript{th} Edition):
\begin{itemize}
\item Chapter 9.1: \emph{Relations and Their Properties}
\item Chapter 9.3: \emph{Representing Relations}
\item Chapter 9.4: \emph{Closures of Relations}
\item Chapter 9.5: \emph{Equivalence Relations}
\item Chapter 9.6: \emph{Partial Orderings}
\end{itemize}

\noteondifficultylevel

\crossline

\begin{enumerate}
\item \strmedium (Rosen 9.1-6)
Determine whether the relation $R$ on the set of real numbers is reflexive,
symmetric, antisymmetric, and/or transitive, where $(x,y)\in R$ if and only if:

\begin{multicols}{4}
\begin{enumerate}
\item $x+y=0$
\item $x=\pm y$
\item $x-y$ is rational
\item $x=2y$
\item $xy\geq 0$
\item $xy= 0$
\item $x= 1$
\item $x= 1$ or $y= 1$
\end{enumerate}
\end{multicols}



\item \strmedium (Rosen 9.1-7)
Determine whether the relation $R$ on the set of all integers
is reflexive, symmetric, antisymmetric, and/or transitive,
where $(x,y)\in R$ if and only if

\begin{multicols}{1}
\begin{enumerate}
\item $x \neq y$.
\item $xy \geq 1$.
\item $x= y+ 1$ or $x = y - 1$.
\item $x= y~(\text{mod }7)$.
\item $x$ is a multiple of $y$.
\item $x$ and $y$ are both negative or both nonnegative.
\item $x= y^{2}$.
\item $x \geq y^{2}$.
\end{enumerate}
\end{multicols}


\item \strmedium  (Rosen 9.1-49) Find the error in the ``proof'' of the following ``theorem.''

``Theorem'': Let $R$ be a relation on a set $A$ that is symmetric and transitive. Then $R$ is reflexive.

``Proof'': Let $a\in A$. Take an element $b\in A$ such that $(a, b) \in R$. Because $R$ is symmetric, we also have $(b, a) \in R$. Now using the transitive property, we can conclude that $(a, a) \in R$ because $(a, b) \in R$ and $(b, a) \in R$.


%\item (Rosen 9.3-15) Let $R$ be the relation represented by the matrix:
%$$
%M_R=\left[ \begin{array}{ccc}
%0&1&0\\
%0&0&1\\
%1&1&0 \end{array} \right]
%$$
%Find the matrices that represent
%\begin{enumerate}
%\item $R^2$
%\item $R^3$
%\item $R^4$
%\end{enumerate}

\item \streasy (Rosen 9.3-19) Draw the directed graphs that represent
the following relations.

\begin{enumerate}
\item ${(1, 2), (1, 3), (1, 4), (2, 3), (2, 4), (3, 4)}$
\item ${(1, 2), (1, 3), (1, 4), (2, 1), (2, 3), (2, 4), (3, 1), (3, 2), (3, 4), (4, 1), (4, 2), (4, 3)}$
\item ${(2, 4), (3, 1), (3, 2), (3, 4)}$
\end{enumerate}


\item \streasy (Rosen 9.4-1) Let $R$ be the relation on the set ${0,1,2,3}$ containing
the ordered pairs $(0,1)$, $(1,1)$, $(1,2)$, $(2,0)$, $(2,2)$, and $(3,0)$.
Find the reflexive closure and the symmetric closure of $R$.

\item \strmedium (Rosen 9.4-25) Use Algorithm 1 from Section 9.4 to find
the transitive closure of the following relations on ${1,2,3,4}$:
\begin{enumerate}

\item ${ (1,2), (2,1), (2,3), (3,4), (4,1) }$

\item $\{ (2,1), (2,3), (3,1), (3,4), (4,1), (4,3) \}$

\end{enumerate}

\item \strmedium (Rosen 9.4-26) Use Algorithm 1 from Section 9.4 to find
the transitive closure of the following relations on ${a,b,c,d,e}$:
\begin{enumerate}
\item $\{(a,c), (b,d), (c,a), (d,b), (e,d)\}$

\item $\{(a,e), (b,a), (b,d), (c,d), (d,a), (d,c), (e,a), (e,b) (e,c), (e,e)\}$
\end{enumerate}

\item  (Rosen 9.4-35)
Show that the closure with respect to the property $P$ of the relation $R = \{(0, 0), (0, 1), (1, 1), (2, 2)\}$ on the set $\{0, 1, 2\}$ does not exist if $P$ is the property:
\begin{itemize}
\item[a)]  \streasy ``is not reflexive.''
\item[b)] \strhard ``has an odd number of elements.''
\end{itemize}

\item \strmedium (Rosen 9.5-1)
Which of these relations on the set $\{0,1,2,3\}$ are equivalence relations?
For the relations that are not equivalence relations, indicate which properties 
of an equivalence relation they do not satisfy.

\begin{enumerate}
\item $\{(0,0),(1,1),(2,2),(3,3)\}$

\item $\{(0,2),(2,0),(2,2),(2,3),(3,2),(3,3)\}$

\item $\{(0,0),(1,1),(1,2),(2,1),(2,2),(3,3)\}$

\item $\{(0,0),(1,1),(1,3),(2,2),(2,3),(3,1),(3,2),(3,3)\}$


\item $\{(0,0),(0,1),(0,2),(1,0),(1,1),(1,2),(2,0),(2,2),(3,3)\}$

\end{enumerate}

\item \strmedium (Rosen 9.5-3)
Which of these relations on the set of all functions from $\mathbb{Z}$ 
to $\mathbb{Z}$ are equivalence relations?
For those that are not equivalence relations, indicate which properties 
they fail to satisfy.

\begin{enumerate}
\item $\{ (f,g) \mid f(1) = g(1) \}$

\item $\{ (f,g) \mid  f(0) = g(0) \text{ or } f(1) = g(1) \}$


\item $\{ (f,g) \mid f(x) - g(x) = 1 \text{ for all } x \in \mathbb{Z} \}$


\item $\{ (f,g) \mid \text{ there exists $c \in \mathbb{Z}$ such that for all $x \in \mathbb{Z}$, } f(x) - g(x) = c \}$


\item $\{ (f,g) \mid f(0) = g(1) \text{ and } f(1) = g(0) \}$

\end{enumerate}

\item \strhard (Rosen 9.5-15) Let $R$ be the relation on ordered pairs of positive 
integers defined by $((a,b),(c,d)) \in R$ if and only if $a+d = b+c$. 
Show that $R$ is an equivalence relation.


\item \strmedium (Rosen 9.6-3) Let $S$ be the set of all people in the world.
Determine whether $(S,R)$ is a poset when $R$ is the relation on $S$ defined by:
\begin{enumerate}
\item $a$ is taller than $b$?


\item $a$ is not taller than $b$?


\item $a=b$ or $a$ is an ancestor of $b$?

\item $a$ and $b$ have a friend in common?

\end{enumerate}

\item \strmedium (Rosen 9.6-4) Let $S$ be the set of all people and $(a,b)\in R$. 
$(S,R)$ is a poset if:
\begin{enumerate}
\item $a$ is not shorter than $b$?


\item $a$ weighs more than $b$?

\item $a=b$ or $a$ is a descendant of $b$?


\item $a$ and $b$ do not have a friend in common?
\end{enumerate}

\item \strmedium (Rosen 9.6-5) Which of the following sets are posets?
\begin{multicols}{3}
\begin{enumerate}
\item $(\mathbb{Z},=)$
\item $(\mathbb{Z},\neq)$
\item $(\mathbb{Z},\geq)$
\item $(\mathbb{Z},\not{\mid})$
\end{enumerate}
\end{multicols}


\item \strmedium (Rosen 9.6-6) Which of the following sets are posets?
\begin{multicols}{4}
\begin{enumerate}
\item $(\mathbb{R},=)$
\item $(\mathbb{R},<)$
\item $(\mathbb{R},\leq)$
\item $(\mathbb{R},\neq)$
\end{enumerate}
\end{multicols}


\item \streasy (Rosen 9.6-17) Determine the lexicographic order of the following $n$-tuples:

\begin{multicols}{3}
\begin{enumerate}
\item $(1,1,2),(1,2,1)$

\item $(0,1,2,3),(0,1,3,2)$

\item $(1,0,1,0,1),(0,1,1,1,0)$
\end{enumerate}
\end{multicols}

\item \strmedium (Rosen 9.6-22) Draw the Hasse diagram for the divisibility relation over:

\begin{multicols}{2}
\begin{enumerate}
\item $\{1,2,3,4,5,6\}$
\item $\{1,3,9,27,81,243\}$
\end{enumerate}
\end{multicols}


\item \strmedium Let $A = \{2, 4\}$ and $B = \{6, 8, 10\}$, and define binary relations $R$ and $S$ as:
\begin{align*}
\forall (x,y) \in A \times B, xRy  &\Leftrightarrow  x \mid y \\
\forall (x,y) \in A \times B, xSy  &\Leftrightarrow  y-4=x
\end{align*}
List the ordered pairs in $A \times B$, $R$, $S$, $R \cup S$, and $R \cap S$.


\item \strhard Let $D$ be a relation on $\mathbb{R}^2$ defined as:
\begin{equation*}
\forall (x,y) \in \mathbb{R}^{2}, xDy  \Leftrightarrow  xy \geq 0.
\end{equation*}
Determine whether $D$ is reflexive, symmetric, and transitive.


\item \label{item:Loureiro-1} \strmedium
\textbf{Composition of relations:} If $R$ is a relation from $A$ to $B$ and $S$ is a relation from $B$ to $C$, the composition $S \circ R$ consists of pairs $(a,c)$ such that there exists $b \in B$ with $(a,b)\in R$ and $(b,c)\in S$.

Find $S \circ R$ where $R: \{1,2,3\} \to \{1,2,3,4\}$ with 
$R = \{ (1,1),(1,4),(2,3),(3,1),(3,4) \}$ and $S: \{1,2,3,4\} \to \{0,1,2\}$ with 
$S = \{ (1,0),(2,0),(3,1),(3,2),(4,1) \}$.


\end{enumerate}

\end{document}

