\input{hmwrk_header.tex}

%====================================================================
% Commands particular to this file
%====================================================================
\usepackage{array}
\newcolumntype{C}[1]{>{\centering\let\newline\\\arraybackslash\hspace{0pt}}m{#1}}
%--------------------------------------------------------------------

\begin{document}

%====================================================================
\homeworktitle{Mathematical induction}{Rosen - Chapter 5}

\crossline

\paragraph{Required reading for this list:}
\emph{Discrete Mathematics and Its Applications} (Rosen, 7\textsuperscript{th} Edition):
\begin{itemize}
\item Chapter 5.1: \emph{Mathematical Induction}
\item Chapter 5.2: \emph{Strong Induction and Well-Ordering}
\end{itemize}

\noteondifficultylevel

\crossline


\begin{enumerate}
\setcounter{enumi}{-1}
\item \textbf{On contradiction and contraposition.}
\begin{enumerate}[(a)]
    \item \streasy Using a proof by contradiction, show that if a function is defined as $f(x) = \frac{2x+3}{x+2}$, then for all positive numbers $x$, $f(x) \neq 2$. Can you also prove this by contraposition?
    \item \streasy Using a proof by contraposition, show that for any natural number $n > 2$, if $2^n - 1$ is prime, then $n$ is odd. Can you also prove it by contradiction?
    \item \strhard Using a proof by contraposition, show that for any natural number $n$, if $2^n-1$ is prime, then $n$ is prime. \textit{(Note: The prime numbers of the form $2^p-1$ for $p$ prime are called Mersenne primes. The largest known prime number is the Mersenne prime $2^{136279841}-1$.)}
    % \item \strmedium Give a proof by contradiction and a proof by contraposition of the following. Given a function $f$ sending integers to integers, if $f(n) = 2n + 3$ for all $n$, then there exists $n,m$ such that $f(n+m) \neq f(n)+f(m)$.
\end{enumerate}
\item (Rosen 5.1-3) Let $P(n)$ be the statement that
$1^{2} + 2^{2} + \cdots + n^{2} = n(n+1)(2n+1)/6$ for the positive integer $n$.

\begin{enumerate}
\item \streasy What is the statement $P(1)$?

\item \streasy Show that $P(1)$ is true by completing the base case.

\item \streasy What is the induction hypothesis?

\item \streasy What do you need to prove in the inductive step?

\item \strmedium Complete the inductive step.

\item \streasy Explain why the above steps show the formula is true for all positive integers $n$.
\end{enumerate}

\item \strmedium (Rosen 5.1-6) Prove that $1 \cdot 1! + 2 \cdot 2! + \cdots + n \cdot n! = (n+1)! -1$, 
for $n \geq 1$.

\item \strmedium (Rosen 5.1-10) Find a formula for 
$\frac{1}{1 \cdot 2} + \frac{1}{2 \cdot 3} + \cdots + \frac{1}{n (n+1)}$ 
by examining small values of $n$ and prove that the formula is correct.

\item \strmedium (Rosen 5.1-11) Find a formula for 
$\frac{1}{2} + \frac{1}{4} + \frac{1}{8} + \cdots + \frac{1}{2^{n}}$ 
by examining small values of $n$ and prove that the formula is correct.

\item \strmedium (Rosen 5.1-21) Prove that $2^n > n^2$ for $n \geq 5$, $n$ integer.

\item \strmedium (Rosen 5.1-33) Prove that $5$ divides
$n^{5}-n$ whenever $n$ is a non-negative integer.

\item \strmedium (Rosen 5.1-60) Prove that 
$\neg{(p_{1} \vee p_{2} \vee \ldots \vee p_{n})}
\equiv 
\neg{p_{1}} \wedge \neg{p_{2}} \wedge \ldots \wedge \neg{p_{n}}$,
for all $n \geq 1$.
(Hint: use De Morgan's law $\neg{(p \vee q)} \equiv \neg{p} \wedge \neg{q}$.)

\item \strmedium What is wrong with this argument by induction: 

``I am going to prove that everyone's eyes are the same color. Ready?

If there is only one person, then it's obviously true; this person's eyes are the same color that this person's eyes.

Suppose it is established that $n-1$ persons must have the same eye color. Consider $n$ persons: the $n-1$ first have the same eye color, and the (n-1) last have the same eye color. Since the two overlap, everyone has the same eye color.

My initialization is verified, and so is my induction. Since I have brown eyes, everyone has brown eyes. Wait a minute, what?''

\item (Rosen 5.2-3) Let $P(n)$ be the proposition ``a postage of $n$ cents can be formed using only 3-cent and 5-cent stamps''. This exercise illustrates a strong induction proof that $P(n)$ is true for $n \geq 8$.

\begin{enumerate}
\item \streasy Show that the propositions $P(8)$, $P(9)$, and $P(10)$ are true, completing the base case.

\item \streasy What is the inductive hypothesis?

\item \streasy What do you need to prove in the inductive step?

\item \strhard Complete the inductive step for $k \geq 10$.
  
\item \strmedium Explain why these steps show that the proposition is true for all $n \geq 8$.
\end{enumerate}

\item (Rosen 5.2-4) Let $P(n)$ be the proposition ``a postage of $n$ cents can be formed using only 4-cent and 7-cent stamps''. This exercise illustrates a strong induction proof that $P(n)$ is true for $n \geq 18$.

\begin{enumerate}
\item \streasy Show that the propositions $P(18)$, $P(19)$, $P(20)$, and $P(21)$ are true, completing the base case.

\item \streasy What is the inductive hypothesis?

\item \streasy What do you need to prove in the inductive step?

\item \strhard Complete the inductive step for $k \geq 21$.
  
\item \strmedium Explain why these steps show that the proposition is true for all $n \geq 18$.
\end{enumerate}

\item \strmedium Consider a chocolate bar made of a single row of $n$ squares as shown below.

\begin{center}
\begin{tabular}{|C{8mm}||C{8mm}||C{8mm}||C{8mm}||C{8mm}||C{8mm}|}
\hline
1 & 2 & 3 & $\cdots$ & $n{-}1$ & $n$ \\
\hline 
\end{tabular}
\end{center}

Suppose you want to separate all the squares of the bar into individual squares. Assume you can only break the bar between two consecutive squares (i.e., you cannot split a square in half, only separate squares from each other).

Using strong induction, prove that for any bar of $n$ squares, exactly $n-1$ breaks are required to separate all squares.

\item \strhard (Rosen 5.2-12) Use strong induction to show that every positive integer $n$ can be written as a sum of distinct powers of 2, i.e., as a sum of a subset of $2^0, 2^1, 2^2, \dots$. 
(Hint: in the inductive step, consider separately the cases $k+1$ odd or even. Note that $\frac{k+1}{2}$ is an integer when $k+1$ is even.)

\item \strmedium The \textit{Fibonacci numbers}, $f_{0},f_{1},\dots$ are defined by $f_{0}=0$, $f_{1}=1$ and $f_{n}=f_{n-1}+f_{n-2}$ for $n=2,3,4,\dots$. Use strong induction to show that
\begin{equation*}
f_n=\frac{1}{\sqrt{5}}\left(\frac{1+\sqrt{5}}{2}\right)^{n}
- \frac{1}{\sqrt{5}}\left(\frac{1-\sqrt{5}}{2}\right)^{n},
\end{equation*}
for $n=0,1,2,\dots$
\end{enumerate}
\end{document}

