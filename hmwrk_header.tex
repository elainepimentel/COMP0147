\documentclass[11pt,a4paper]{article}

%====================================================================
% LaTeX Packages
%====================================================================
\usepackage{geometry}
\usepackage{enumerate}
\usepackage{mdwlist} % Allows for suspeding and resuming lists.
\geometry{a4paper,left=1.5cm,right=1.5cm,top=2.5cm,bottom=2cm}
\usepackage{epsf}
\usepackage{amsmath,amssymb}
\usepackage{array} % Allows fixing row width.
\usepackage{amsthm} % Allows for \qed symbol
\usepackage{graphicx}
\usepackage{nicefrac}
\usepackage{multicol}
\usepackage{xspace} % Introduces spaces after macros only when needed.
%--------------------------------------------------------------------


% Information about the course.
\newcommand{\strcourse}{Discrete Mathematics - COMP0147}
\newcommand{\strinstitution}{\includegraphics[height=1cm]{UCL-logo.jpeg}}
\newcommand{\strsemester}{2025 - Term 1}

% Information about the instructor.
\newcommand{\strinstructor}{Prof. Elaine Pimentel}
\newcommand{\stremail}{e.pimentel@ucl.ac.uk}

% Information about assigment
\newcommand{\strtypeofassignment}{List of exercises}
\newcommand{\strsolution}{(Solution)}
\newcommand{\strinstructorsolution}{Solution}
\newcommand{\streasy}{\level{Easy}}
\newcommand{\strmedium}{\level{Medium}}
\newcommand{\strhard}{\level{Hard}}
\newcommand{\strnoteondifficultylevel}{
\paragraph{Note:} 
The exercises are classified into difficulty levels:
easy, medium, and hard.
This classification, however, is only indicative.
Different people may disagree about the difficulty level of the same exercise.
Do not be discouraged if you see a difficult exercise--you may find that it is actually easy, by discovering a simpler way to solve it!
}
%--------------------------------------------------------------------

\newcommand{\level}[1]{\textbf{\texttt{[#1]}}\xspace}
\newcommand{\qm}[1]{``#1''}
\newcommand{\prop}[1]{\emph{\qm{#1}}} % Proposition in natural language
\newcommand{\lazyfrac}[2]{#1/#2}
\newcommand{\crossline}{\noindent\makebox[\linewidth]{\rule{\textwidth}{1pt}}}
\newcommand{\ie}{{\em i.e.}}
\newcommand{\eg}{{\em e.g.}}

\newcommand{\imp}{\rightarrow} % Implication symbol
\newcommand{\doubleimp}{\leftrightarrow} % Double implication symbol

\newcommand{\floor}[1]{\left\lfloor #1 \right\rfloor}
\newcommand{\ceil}[1]{\left\lceil #1 \right\rceil}

\newcommand{\green}[1]{{\color{green} #1}}
\newcommand{\red}[1]{{\color{red} #1}}


\usepackage{pifont}
\newcommand{\cmark}{\ding{51}}
\newcommand{\xmark}{\ding{55}}

\newcommand{\yes}{\green{\cmark}}
\newcommand{\no}{\red{\xmark}}


\usepackage{tikz}
\newcommand*\circled[1]{\tikz[baseline=(char.base)]{
            \node[shape=circle,draw,inner sep=2pt] (char) {#1};}}
\newcommand{\orgcellA}[3]{\begin{Large}$({#1},{#2})$\end{Large} {\circled{#3}}}
\newcommand{\orgcellB}[2]{\begin{Large}$({#1},{#2})$\end{Large} {$\ldots$}}

% We need this in order to be able to print both a handouts version
% and a solution version of the homework.
\ifdefined\hidesolution
	\newcommand{\solution}[1]{}  
\else
	\newcommand{\solution}[1]{\paragraph{\strinstructorsolution:} #1 \vspace{4mm}}
\fi

\newcommand{\homeworktitle}[2]{
\begin{center}
\begin{flushleft}
\noindent \textbf{\strinstitution \hfill \strsemester} \\
\textbf{\strcourse \hfill \strinstructor}
\end{flushleft} 
\ \\
\textbf{\MakeUppercase{\strtypeofassignment}}\\
\textsc{#1\\
(#2)}
\end{center}
}

\newcommand{\noteondifficultylevel}{
\strnoteondifficultylevel
}
%--------------------------------------------------------------------





