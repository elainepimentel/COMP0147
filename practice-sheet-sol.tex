\documentclass[11pt,a4]{exam}

\usepackage[colorlinks=true, allcolors=blue]{hyperref}
\usepackage{listings}
\usepackage{multicol}
\usepackage{enumerate}
\usepackage{qtree}
\usepackage{xcolor}
\usepackage{proof}
\usepackage{graphicx}
\usepackage{amsmath}
\usepackage{booktabs}
\usepackage{amsthm}
\usepackage{latexsym}
\usepackage{amsmath}
\usepackage{amssymb}
\usepackage{amsfonts}
\usepackage{graphicx}
\usepackage{xspace}

\usepackage{tikz}
\usetikzlibrary{positioning,arrows.meta}

\def\cC{\mathcal{C}}
\def\cV{\mathcal{V}}
\def\cP{\mathcal{P}}

\newcommand\lra{\longrightarrow}
\newcommand\lrai{\Longrightarrow}
\newcommand\iimp{\supset}

\newcommand{\Frame}[1]{
    \fbox{
      \parbox[c]{\textwidth}
      {
       \noindent #1
      }
    }
}

\renewcommand{\solution}[1]{\paragraph{Solution:} #1 \vspace{4mm}}

\usepackage{draftwatermark}
\SetWatermarkAngle{-30}
\SetWatermarkText{\includegraphics{answers}}

\usepackage[centerlast,small,sc]{caption}
\newcommand{\nat}{N}
\setlength{\captionmargin}{60pt}
\boxedpoints
\pointpoints{Mark}{Marks}
\addpoints
\hqword{Question}
\hpword{Mark}
\hsword{Cal.}
\setlength\linefillheight{.5cm}
%\printanswers
\extraheadheight[0.5in]{0.2in}


\newcommand{\course}{COMP0147: Discrete Mathematics \\ 25/26}
\newcommand{\subject}{Practice sheet}
\newcommand{\dt}{Nov/2025}


\firstpageheader{\includegraphics[height=1.3cm]{UCL-logo.jpeg}
}%
{\large\bf  \course}{\bf \subject \\ \dt}

\runningheader{
Name:\\
}{}{}
\firstpagefooter{}{}{}
\runningfooter{}{Page \thepage}{}
\runningheadrule
\footrule
\lfoot{}
\cfoot{Page  \thepage\ of \numpages}
\rfoot{}


\begin{document}

\hbox to \textwidth{Name and ID:\enspace\hrulefill}
%\vspace{1cm}
%\hbox to \textwidth{Group:\enspace\hrulefill}



\begin{center}
\Frame{
\begin{multicols}{2}
\scriptsize
\begin{enumerate}
\item {\em This is not a mock exam!} It is a set of selected exercises covering all the material in the module.
\item Try to solve all the questions and keep track of how long you take -- it should be around 2 hours.
\item The sample is marked out of 100 in total. The number of marks for each question roughly corresponds to the difficulty.
\item Good practice!
\end{enumerate}
\end{multicols}
}
\end{center}


\lstset{
	tabsize = 1, %% set tab space width
	showstringspaces = false, %% prevent space marking in strings, string is defined as the text that is generally printed directly to the console
	numbers = left, %% display line numbers on the left
	commentstyle = \color{green}, %% set comment color
	keywordstyle = \color{blue}, %% set keyword color
	stringstyle = \color{red}, %% set string color
	rulecolor = \color{black}, %% set frame color to avoid being affected by text color
	basicstyle = \fontsize{7}{10} \ttfamily , %% set listing font and size
	breaklines = true, %% enable line breaking
	numberstyle = \tiny,
	xleftmargin=2.5em,frame=single,framexleftmargin=2.5em
  % literate={\ \ }{{\ }}1
}



\begin{questions}


\question[6] 
Determine whether each of the arguments below is correct or incorrect, and explain why.
You must formalize the argument using propositions/predicates and indicate which rules of inference were used.

\begin{enumerate}

\item[a.] It is fun to drive convertible cars.
Claudia's car is not a convertible.
Therefore, Claudia's car is not fun to drive.

\item[b.] Alice is taking COMP0147 or COMP003.
Alice is not taking COMP0147 or she is taking Calculus.
Therefore, Alice is taking COMP0003 or Calculus.

\end{enumerate}
\solution{

\begin{itemize}
\item[a.] Notation for the predicate symbols:
    \[
    \mathrm{Conv}(x) : \text{$x$ is a convertible}, \quad 
    \mathrm{Fun}(x) : \text{$x$ is fun to drive}
    \]
    and a constant \(c\) for ``Claudia's car.'' 
\[
\begin{array}{l}
\text{Premise 1: It is fun to drive convertible cars.} \\
\quad (\forall x)(\mathrm{Conv}(x) \to \mathrm{Fun}(x)) \\[2mm]
\text{Premise 2: Claudia's car is not a convertible.} \\
\quad \neg \mathrm{Conv}(c) \\[2mm]
\text{Conclusion: Claudia's car is not fun to drive.} \\
\quad \neg \mathrm{Fun}(c)
\end{array}
\]

The statement is false.

\textbf{Explanation:} By \emph{Universal Instantiation} we obtain:
\[
\mathrm{Conv}(c) \to \mathrm{Fun}(c)
\]
but from \(\mathrm{Conv}(c) \to \mathrm{Fun}(c)\) and \(\neg \mathrm{Conv}(c)\) we \emph{cannot} deduce \(\neg \mathrm{Fun}(c)\). This is the fallacy of \emph{denying the antecedent}.  

\textbf{Countermodel:} Let \(\mathrm{Conv}(c) = \text{false}\), \(\mathrm{Fun}(c) = \text{true}\). Then both premises are true but the conclusion is false, showing invalidity.

\item[b.] Notation for the propositional letters:
    \[
    P : \text{Alice takes COMP0147}, \quad 
    Q : \text{Alice takes COMP003}, \quad 
    R : \text{Alice takes Calculus}.
    \]

\[
\begin{array}{l}
\text{Premise 1: Alice is taking COMP0147 or COMP003.} \\
\quad P \vee Q \\[1mm]
\text{Premise 2: Alice is not taking COMP0147 or she is taking Calculus.} \\
\quad \neg P \vee R \\[1mm]
\text{Conclusion: Alice is taking COMP003 or Calculus.} \\
\quad Q \vee R
\end{array}
\]

The statement is true.

\textbf{Explanation:} This follows by the \emph{resolution rule}:
\[
(P \vee Q), (\neg P \vee R) \vdash Q \vee R
\]
\emph{Proof by cases:}
\begin{itemize}
    \item Case 1: \(P\) is true. Then \(\neg P \vee R\) forces \(R\) to be true, so \(Q \vee R\) holds.
    \item Case 2: \(P\) is false. Then \(P \vee Q\) forces \(Q\) to be true, so \(Q \vee R\) holds.
\end{itemize}
In both cases, the conclusion is true.
\end{itemize}
}

\question[4] Let $m_1 , m_2 , \ldots , m_n$ be pairwise relatively prime integers greater than or equal to 2. Show that if $a \equiv b\pmod{m_i}$ for $i = 1,2,\ldots,n$, then $a \equiv b \pmod{m}$, where $m = m_1\cdot m_2\cdot\ldots \cdot m_n$.

\solution{

\begin{itemize}
\item Using factorisation. The premise is $a \equiv b \pmod{m_i}$ for all $i = 1, \ldots, n$.
By the definition of congruence, this means that $m_i$ divides the difference $(a - b)$ for every $i$:
\[
    m_i \mid (a - b) \quad \text{for all } i = 1, 2, \ldots, n.
\]
We want to show that $m = m_1 m_2 \cdots m_n$ divides $(a-b)$, or $m \mid (a-b)$.

Since $m_1, m_2, \ldots, m_n$ are pairwise relatively prime, they share no prime factors.
Let the prime factorization of each $m_i$ be:
\[
    m_i = p_{i,1}^{e_{i,1}} p_{i,2}^{e_{i,2}} \cdots p_{i,k_i}^{e_{i,k_i}}
\]
where $p_{i,j}$ are distinct prime numbers for different values of $i$.
The prime factorization of the product $m = m_1 m_2 \cdots m_n$ is simply the union of all these distinct prime power factors:
\[
    m = \prod_{i=1}^{n} m_i = \left( \prod_{j=1}^{k_1} p_{1,j}^{e_{1,j}} \right) \left( \prod_{j=1}^{k_2} p_{2,j}^{e_{2,j}} \right) \cdots \left( \prod_{j=1}^{k_n} p_{n,j}^{e_{n,j}} \right)
\]
In other words, the set of all prime power factors of $m$ is exactly the union of the prime power factors of $m_1, m_2, \ldots, m_n$.

From the initial premise, we know $m_i \mid (a-b)$.
This means that for every $i$, all prime factors of $m_i$, with their correct exponents, must also be factors of the difference $(a-b)$.
Specifically, for a prime power $p^e$ that is a factor of $m_i$ (i.e., $p^e \mid m_i$), we must have $p^e \mid (a-b)$.

Let $P$ be the set of all distinct prime power factors $p^e$ of $m$.
\[
    P = \{ p_{i,j}^{e_{i,j}} \mid i=1,\ldots,n; j=1,\ldots,k_i \}
\]
From Step 1, we established that $m = \prod_{p^e \in P} p^e$.
From Step 2, we know that every prime power factor $p^e \in P$ must divide $(a-b)$.

A key theorem in number theory states that if a set of pairwise relatively prime numbers (in this case, the set of distinct prime power factors $P$) all divide an integer $(a-b)$, then their product must also divide $(a-b)$.
Since the product of all prime power factors in $P$ is exactly $m$, we conclude:
\[
    m \mid (a - b)
\]
By the definition of modular congruence, this implies:
\[
    a \equiv b \pmod{m}
\]
The proof is complete.

\item Using induction.
%\textbf{Problem:} Let $m_1, m_2, \ldots, m_n$ be pairwise relatively prime integers greater than or equal to 2. Show that if $a \equiv b \pmod{m_i}$ for $i = 1, 2, \ldots, n$, then $a \equiv b \pmod{m}$, where $m = m_1 m_2 \cdots m_n$.
%
We are given the set of congruences:
\[
    a \equiv b \pmod{m_i} \quad \text{for all } i = 1, 2, \ldots, n.
\]
By the definition of modular congruence, this means that $m_i$ divides the difference $(a-b)$ for each $i$:
\[
    m_i \mid (a - b) \quad \text{for all } i = 1, 2, \ldots, n.
\]
This implies that for each $i$, there exists an integer $k_i$ such that:
\[
    a - b = k_i m_i
\]

Since $m_1 \mid (a-b)$ and $m_2 \mid (a-b)$, and we are given that $m_1$ and $m_2$ are relatively prime (i.e., $\gcd(m_1, m_2) = 1$), it follows from a fundamental property of divisibility that their product must also divide $(a-b)$.
\[
    m_1 m_2 \mid (a - b)
\]

We can formally prove this by induction on the number of moduli $n$.

\subsection*{Base Case (n=2)}
Assume $m_1 \mid (a-b)$ and $m_2 \mid (a-b)$, with $\gcd(m_1, m_2) = 1$.
Since $m_1 \mid (a-b)$, we have $a-b = k m_1$ for some integer $k$.
Substituting this into the second divisibility relation, we get $m_2 \mid k m_1$.
Since $\gcd(m_1, m_2) = 1$, we must have $m_2 \mid k$.
Thus, $k = l m_2$ for some integer $l$.
Substituting this back into the expression for $(a-b)$:
\[
    a - b = (l m_2) m_1 = l (m_1 m_2)
\]
This shows that $m_1 m_2 \mid (a - b)$, which proves the base case.

\subsection*{Inductive Step}
Let $P(k)$ be the proposition that if $a \equiv b \pmod{m_i}$ for $i=1, \ldots, k$, and $m_1, \ldots, m_k$ are pairwise relatively prime, then $a \equiv b \pmod{M_k}$, where $M_k = m_1 m_2 \cdots m_k$.

\textbf{Inductive Hypothesis (I.H.):} Assume $P(n-1)$ holds for $n-1 \geq 2$. That is, we assume:
\[
    a \equiv b \pmod{m_1 m_2 \cdots m_{n-1}}
\]
Let $M_{n-1} = m_1 m_2 \cdots m_{n-1}$. By the I.H., we have $M_{n-1} \mid (a-b)$.
We are also given the congruence for $m_n$:
\[
    m_n \mid (a - b)
\]
Since $m_1, \ldots, m_n$ are pairwise relatively prime, $m_n$ is relatively prime to every factor in $M_{n-1}$. Therefore, $m_n$ is relatively prime to their product $M_{n-1}$:
\[
    \gcd(M_{n-1}, m_n) = 1
\]
Now we apply the result from the base case (or the property used above) to the two moduli $M_{n-1}$ and $m_n$.
Since $M_{n-1} \mid (a-b)$, $m_n \mid (a-b)$, and $\gcd(M_{n-1}, m_n) = 1$, it must be true that their product divides $(a-b)$:
\[
    M_{n-1} m_n \mid (a - b)
\]
Since $M_{n-1} m_n = m_1 m_2 \cdots m_n = m$, we have:
\[
    m \mid (a - b)
\]
By the definition of modular congruence, this is equivalent to:
\[
    a \equiv b \pmod{m}
\]
Thus, the proposition holds for $n$.

By the principle of induction, the result holds for all $n \geq 2$.

\end{itemize}
}

\question[10]
The {\em symmetric difference} of sets $A$ and $B$, denoted by $A \oplus B$, is
the set containing those elements in either $A$ or $B$, but not in both $A$ and $B$.
\begin{enumerate}
    \item[a.] What can you say about the sets $A$ and $B$ if $A\oplus B=A$?
    \item[b.] Suppose that $A, B, C$ are sets such that $A\oplus C = B \oplus C$. Must it be the case that $A = B$?
\end{enumerate}    
\solution{
First of all, observe that: $A\oplus B = (A- B)\cup(B- A).$

\begin{enumerate}
    \item[a.] Suppose that $A \oplus B = A$, that is,
    \[
    A = (A -B)\cup(B-A).
    \]
    Suppose that there exists $b\in B$. If $b\in A$, then $b$ is not $A \oplus B$, so $A \oplus B\not= A$. On the other hand, if $b \in $A, then it must be the case that $b\in A\oplus B$, so again $A\oplus B \not= A$. Thus in either case, $A \oplus B \not= A$. We conclude that $B$ cannot have any elements, hence $B=\varnothing$.
   

    \item[b.] Yes. Assume that $A \oplus C = B \oplus C$, and let $x \in A$.  There are two possibilities:
    \begin{itemize}
    \item $x \in C$: by definition $x \notin A \oplus C$. Therefore $x \notin B \oplus C$. Now if $x\notin B$, then $x \in B \oplus C$ (since $x \in C$ by assumption). Since this is not true, we conclude that $x \in B$, as desired. 
    \item $x \notin C$: then $x \in A \oplus C$. Therefore $x \in B \oplus C$ as well. Again, if $x \notin B$, then $x\notin B \oplus C$ (since $x \notin C$ by assumption). Once again we conclude that $x \in B$.
    \end{itemize}
    In any case, if $x\in A$ then $x\in B$. Since the argument is symmetric, the proof is complete.
%    Suppose that $A\oplus C = B\oplus C$. Using the fact that
%    \[
%    (X\oplus Y)\oplus Y = X,
%    \]
%    we apply $\oplus C$ to both sides:
%    \[
%    (A\oplus C)\oplus C = (B\oplus C)\oplus C.
%    \]
%    Therefore $A = B$.
\end{enumerate}

}

\question[15] Consider a function $f : A \to B$.

\begin{itemize}
    \item 
    A function $r : B \to A$ is called a \emph{right inverse} of $f$ if
    \[
    f \circ r = \mathrm{id}_B.
    \]
    This means that for every $b \in B$, $f(r(b)) = b$.

    \item 
    A function $\ell : B \to A$ is called a \emph{left inverse} of $f$ if
    \[
    \ell \circ f = \mathrm{id}_A.
    \]
    This means that for every $a \in A$, $\ell(f(a)) = a$.
\end{itemize}
\begin{enumerate}
 \item[a.] Prove  that a function $h: A\to B$,
\begin{itemize}
  \item[i.] has a right inverse iff $h$ is surjective;
  \item[ii.] has a left inverse  iff $h$ is injective;
  \item[iii.] is invertible iff $h$ is a bijection.
\end{itemize}

    \item[b.] Let the function $f:\mathbb{R}\to \mathbb{R}$ be defined as:
    \[
    f(x)=\left\{\begin{array}{ll}
    \sqrt{x} & \mbox{ if } x>0,\\
    0 & \mbox{ otherwise.}
    \end{array}
    \right.
    \]
    \begin{enumerate}
    \item[i.] Does $f$ have a right inverse? If so, give an example of a right inverse.
    \item[ii.] Does $f$ have a left inverse? If so, give an example of a left inverse.
    \item[iii.] Is $f$ invertible? If it is, what is its inverse?
    \end{enumerate}

    \item[c.] Let the function $g:\mathbb{R}\to [0,\infty)$ be defined as:
    \[
    g(x)=\left\{\begin{array}{ll}
    \sqrt{x} & \mbox{ if } x>0,\\
    0 & \mbox{ otherwise.}
    \end{array}
    \right.
    \]
    \begin{enumerate}
    \item[i.] Does $g$ have a right inverse? If so, give an example of a right inverse.
    \item[ii.] Does $g$ have a left inverse? If so, give an example of a left inverse.
    \item[iii.] Is $g$ invertible? If it is, what is its inverse?
    \end{enumerate}
\end{enumerate}

\solution{
\begin{itemize}
\item[a.]
\begin{itemize}
\item[i.] Suppose \(h\) has a right inverse; that is, there exists \(r:B\to A\) with
\[
h\circ r=\mathrm{id}_B.
\]
For any \(b\in B\) we have \(h(r(b))=b\), so every \(b\in B\) has a preimage under \(h\).
Hence \(h\) is surjective.

Conversely, suppose \(h\) is surjective. For each \(b\in B\) the nonempty set
\(h^{-1}(\{b\})\subseteq A\) of preimages of \(b\) is nonempty. Choosing one element
\(a_b\in h^{-1}(\{b\})\) for each \(b\in B\) and defining \(r(b):=a_b\) yields a function
\(r:B\to A\) with \(h(r(b))=b\) for all \(b\), i.e. \(h\circ r=\mathrm{id}_B\). 

%\emph{Remark:} the construction above uses a choice of one preimage for each \(b\in B\).
%Thus the existence of a right inverse for every surjection is guaranteed if one assumes
%the Axiom of Choice. The implication ``right inverse \(\Rightarrow\) surjective'' holds
%without any choice assumption.

\item [ii.] Suppose \(h\) has a left inverse; that is, there exists \(\ell:B\to A\) with
\[
\ell\circ h=\mathrm{id}_A.
\]
If \(h(a_1)=h(a_2)\) then applying \(\ell\) to both sides gives
\[
a_1=(\ell\circ h)(a_1)=(\ell\circ h)(a_2)=a_2,
\]
so \(h\) is injective.

Conversely, suppose \(h\) is injective. The restriction \(h: A\to h(A)\)
is a bijection onto its image \(h(A)\), hence it has an inverse
\(\tilde{\ell}:h(A)\to A\) satisfying \(\tilde{\ell}\circ h=\mathrm{id}_A\). To obtain
a function \(\ell:B\to A\) defined on all of \(B\) we may extend \(\tilde{\ell}\) arbitrarily
to the points of \(B- h(A)\): choose for each \(b\in B- h(A)\) some
element \(a_0\in A\) (if \(A\) is nonempty) and set \(\ell(b):=a_0\). Then \(\ell\) satisfies
\(\ell\circ h=\mathrm{id}_A\), so \(\ell\) is a left inverse of \(h\).

%\emph{Remark on edge cases:} if \(A=\varnothing\) but \(B\neq\varnothing\), then there
%is no function \(B\to A\), so no left inverse can exist despite \(h\) being injective
%(vacuously). Apart from this trivial pathological case, the construction above provides
%a left inverse without any appeal to the Axiom of Choice (the values on \(B- h(A)\)
%can all be taken equal to any fixed element of \(A\), which requires that \(A\) be nonempty).
%In particular, if \(A\) is nonempty and \(h\) is injective then \(h\) has a left inverse.

\item[iii.] Suppose \(h\) is invertible: there exists \(g:B\to A\) with
\[
g\circ h=\mathrm{id}_A\qquad\text{and}\qquad h\circ g=\mathrm{id}_B.
\]
Then from part (2) \(h\) is injective, and from part (1) \(h\) is surjective. Hence \(h\)
is a bijection.

Conversely, if \(h\) is a bijection then it has an inverse map
\(h^{-1}:B\to A\) which is both a left and a right inverse, i.e.
\[
h^{-1}\circ h=\mathrm{id}_A,\qquad h\circ h^{-1}=\mathrm{id}_B.
\]
Thus \(h\) is invertible.
\end{itemize}
\item[b.] The function $f$ has image (range) $\operatorname{Im}f=[0,\infty)$.

\begin{enumerate}
\item[i.] \emph{Right inverse?} \\
A right inverse $r:\mathbb{R}\to\mathbb{R}$ would satisfy $f(r(y))=y$ for all $y\in\mathbb{R}$. But $f$ never attains negative values, so no negative $y$ can be written as $f(x)$. Hence there is no $r$ with $f(r(y))=y$ for every $y\in\mathbb{R}$. Therefore \emph{$f$ has no right inverse}.

\item[ii.] \emph{Left inverse?} \\
A left inverse $\ell:\mathbb{R}\to\mathbb{R}$ must satisfy $\ell(f(x))=x$ for all $x\in\mathbb{R}$. This would force $\ell(0)=x$ for every $x\le0$ (since $f(x)=0$ for all $x\le0$), which is impossible. Thus $f$ is not injective and \emph{has no left inverse}.

\item[iii.] \emph{Inverse?} \\
The inverse exists only for bijections. Since $f$ is neither injective nor surjective (onto $\mathbb{R}$), \emph{no two-sided inverse exists}.
\end{enumerate}

\item[c.] The function $g$
has the same rule as $f$ but its codomain is $[0,\infty)$. Its image is $\operatorname{Im}g=[0,\infty)$, so $g$ is \emph{surjective} onto its codomain.

\begin{enumerate}
\item[i.] \emph{Right inverse?} \\
Yes. For example define $r:[0,\infty)\to\mathbb{R}$ by
\[
r(y)=y^2\qquad (y\ge0).
\]
Then for every $y\in[0,\infty)$,
\[
g(r(y))=g(y^2)=\sqrt{y^2}=y,
\]
so $g\circ r=\mathrm{id}_{[0,\infty)}$. Thus $r$ is a right inverse of $g$.
(There are many right inverses: for $y>0$ one must choose a preimage $x$ with $x=y^2$, and for $y=0$ any $x\le0$ or $x=0$ will do; $r(y)=y^2$ is the natural choice.)

\item[ii.] \emph{Left inverse?} \\
No. A left inverse $\ell:[0,\infty)\to\mathbb{R}$ would require $\ell(g(x))=x$ for all $x\in\mathbb{R}$. But $g(x)=0$ for all $x\le0$, so this would force $\ell(0)=x$ for every $x\le0$, impossible. Hence $g$ is not injective and \emph{has no left inverse}.

\item[iii.] \emph{Inverse?} \\
No, because $g$ is not injective (many $x\le0$ map to $0$), so it is not bijective and there is no two-sided inverse.
\end{enumerate}
\end{itemize}
}



\question[10]
Prove by mathematical induction that for all integers $n \geq 1$,
\[
1^3 + 2^3 + \dots + n^3 = (1 + 2 + \dots + n)^2.
\]

\solution{
\textbf{Base case:} \(n=1\). Left-hand side \(=1^3=1\). Right-hand side \(=(1)^2=1\). So the statement holds for \(n=1\).

\textbf{Inductive step:} Assume the identity holds for some \(n\ge1\); that is assume
\[
1^3+2^3+\cdots+n^3=\bigl(1+2+\cdots+n\bigr)^2.
\]
Let \(S_n=1+2+\cdots+n\). Using the induction hypothesis,
\[
1^3+2^3+\cdots+n^3+ (n+1)^3 = S_n^2 + (n+1)^3.
\]
We must show the right-hand side equals \((S_n+(n+1))^2\). Indeed,
\[
(S_n+(n+1))^2 - S_n^2 = 2S_n(n+1) + (n+1)^2.
\]
But \(S_n=\dfrac{n(n+1)}{2}\), so
\[
2S_n(n+1) + (n+1)^2
= 2\cdot\frac{n(n+1)}{2}\,(n+1) + (n+1)^2
= n(n+1)^2 + (n+1)^2
= (n+1)^2(n+1)
= (n+1)^3.
\]
Hence
\[
S_n^2 + (n+1)^3 = (S_n+(n+1))^2,
\]
which shows
\[
1^3+2^3+\cdots+n^3+(n+1)^3 = \bigl(1+2+\cdots+n+(n+1)\bigr)^2.
\]
Thus the identity holds for \(n+1\). By the principle of mathematical induction the formula is true for all integers \(n\ge1\). \(\square\)
}


\question[10] Prove that at a party where there are at least two people, there are two people who know the same number of other people there.
\solution{
Label the people \(P_1,\dots,P_n\). For each \(i\) let \(d_i\) denote the number of other people that \(P_i\) knows.
Clearly \(d_i\in\{0,1,\dots,n-1\}\) for every \(i\).

If some person knows nobody, say for some \(i\) we have \(d_i=0\), then no person can know everyone else,
so no one can have degree \(n-1\). Thus the possible degree values actually realized lie in the set
\(\{0,1,\dots,n-2\}\), which has \(n-1\) elements.

If no person has degree \(0\), then every person knows at least one other person, so degrees lie in
\(\{1,2,\dots,n-1\}\), which also has \(n-1\) elements.

Therefore, in either case the \(n\) people have degrees that take values from a set of size \(n-1\).
By the pigeonhole principle (or simply by counting), two of the \(d_i\)'s must be equal. Hence there are
two people who know the same number of other people.
}


\question[5]
A survey of 100 students found that:
\begin{itemize}
    \item 60 study Mathematics,
    \item 45 study Computer Science,
    \item 30 study both.
\end{itemize}
\begin{enumerate}
    \item[a.] How many students study at least one of the two subjects?
    \item[b.] How many study only Mathematics?
    \item[c.] How many study neither subject?
\end{enumerate}
Express your results using set operations and the inclusion-exclusion principle.

\solution{
Let \(M\) be the set of students who study Mathematics and \(C\) the set of students who study Computer Science.
We are given
\[
|M|=60,\qquad |C|=45,\qquad |M\cap C|=30,\qquad \text{total students }=100.
\]

\begin{enumerate}
\item[a.] Number who study at least one of the two subjects.\\
By the inclusion--exclusion principle,
\[
|M\cup C|=|M|+|C|-|M\cap C|=60+45-30=75.
\]

\item[b.] Number who study only Mathematics.\\
Students who study only Mathematics are those in \(M\) but not in \(C\), i.e. \(M- C\). Hence
\[
|M- C|=|M|-|M\cap C|=60-30=30.
\]

\item[c.] Number who study neither subject.\\
Those who study neither are the complement of \(M\cup C\) (inside the 100 students), so
\[
\text{neither} = 100-|M\cup C|=100-75=25.
\]
\end{enumerate}
}

\question[10] Let $X=\{(a,b)\mid a,b\in\mathbb{Z}, b\not=0\}$ and define a relation $R$ on $X$ by
\[
((a,b),(c,d))\in R \qquad\Longleftrightarrow\qquad ad=bc.
\]
Show that $R$ is an equivalence relation.
\solution{
\begin{itemize}
\item Reflexive.
For any $(a,b)\in X$ we have $ab=ba$, so $((a,b),(a,b))\in R$. Thus $R$ is reflexive.

\item Symmetric.
If $((a,b),(c,d))\in R$ then $ad=bc$. Rearranging gives $cb=da$, so $((c,d),(a,b))\in R$. Hence $R$ is symmetric.

\item Transitive.
Suppose $((a,b),(c,d))\in R$ and $((c,d),(e,f))\in R$. Then
\[
ad=bc \qquad\text{and}\qquad cf=de.
\]
Multiply the first equation by $f$ and the second by $b$ to obtain
\[
adf = bcf \qquad\text{and}\qquad bcf = bde.
\]
Thus $adf=bde$. Since $d\not=0$, we may divide both sides by $d$ to get $af=be$, i.e.
$((a,b),(e,f))\in R$. Therefore $R$ is transitive.
\end{itemize}

Since $R$ is reflexive, symmetric and transitive, it is an equivalence relation.

{\em Remark.}
The equivalence class of $(a,b)$ is the set
\[
\{(c,d)\in X \mid ad=bc\},
\]
so equivalence classes correspond exactly to the rational numbers $\dfrac{a}{b}$ (written in any representative form).
}

\question[20] A poset $(R, \preceq)$ is {\em well-founded} if there is no infinite decreasing sequence of elements in the poset, that is, elements $x_1,x_2,\ldots,x_n,\ldots$ such that $\cdots \prec x_n \prec \cdots \prec x_2 \prec x_1$. A poset $(R, \preceq)$ is {\em dense} if for all $x ,y \in S$ with $x \prec y$, there is an element $z \in R$ such that $x \prec z \prec y$.

Let $\Sigma=\{\text{the 26 lowercase English letters}\}$ and let $\Sigma^{*}$ be the set of all finite strings over $\Sigma$.  Equip $\Sigma^{*}$ with the usual lexicographic order $\leq$ defined as follows: for $x,y\in\Sigma^*$, write $x<y$ iff either
\begin{enumerate}
  \item $x$ is a proper prefix of $y$, or
  \item at the first position where $x$ and $y$ differ, the letter of $x$ is less than the letter of $y$ (according to the usual order $a<b<\cdots<z$).
\end{enumerate}

Show that $(\Sigma^*,\leq)$ is
\begin{itemize}
\item [a)] not well-founded;
\item[b)] not dense.
\end{itemize}
\solution{
\begin{itemize}
\item[a)] Not well-founded.
Consider the infinite sequence
\[
s_0 = \texttt{b},\qquad s_1=\texttt{ab},\qquad s_2=\texttt{aab},\qquad s_3=\texttt{aaab},\qquad\ldots
\]
in which $s_n$ is the string consisting of $n$ copies of \texttt{a} followed by \texttt{b}.  For every $n\ge 0$ we have $s_n>s_{n+1}$: indeed, compare $s_n$ and $s_{n+1}$; they agree on the first $n$ letters (all \texttt{a}), and at the $(n+1)$-st position $s_n$ has \texttt{b} while $s_{n+1}$ has \texttt{a}, and since \texttt{b}>\texttt{a} we get $s_n>s_{n+1}$.  Thus
\[
\texttt{b}>\texttt{ab}>\texttt{aab}>\texttt{aaab}>\cdots
\]
is an infinite strictly decreasing chain, so $(\Sigma^*,<)$ is not well-founded.

\item[b)] Not dense.
Take $x=\texttt{a}$ and $y=\texttt{aa}$. Clearly $x<y$ because $x$ is a proper prefix of $y$. Suppose $z\in\Sigma^*$ satisfies $x<z<y$.  Since $x=\texttt{a}$ is a prefix of $y=\texttt{aa}$, any $z$ with $x<z$ must begin with the letter \texttt{a}. Write $z=\texttt{a}w$ for some (possibly empty) string $w$. The condition $z<\texttt{aa}$ forces $w<\texttt{a}$ in the lexicographic order of strings (compare $w$ with the one-letter string \texttt{a} at the first position where they differ, or note that if $w$ is empty then $z=\texttt{a}=x$). But there is no nonempty string $w$ with $w<\texttt{a}$, and $w$ cannot be empty because that would make $z=x$. Hence no such $z$ exists. Therefore there are $x<y$ with no $z$ strictly between them, so the order is not dense.
\end{itemize}

}



\question[10] The {\em line graph} $L(G)$ of a simple graph $G$ is the graph whose vertices are in one-to-one correspondence with the edges of $G$. Two vertices of $L(G)$ are adjacent if and only if the corresponding edges in G have an end-point in common.
\begin{itemize}
\item[a.] Determine the line graphs of the two graphs below.
\begin{center}
\begin{tikzpicture}[vertex/.style={draw,circle,inner sep=1.5pt,minimum size=12pt}, >=Stealth]
% G1 triangle
\node[vertex] (A) at (0,1) { };
\node[vertex] (B) at (-0.87,-0.5) { };
\node[vertex] (C) at (0.87,-0.5) { };
\draw (A) -- (B) -- (C) -- (A);
\node at (0,-1.0) {$G_1$};
\begin{scope}[xshift=4.5cm]
% G2 K1,3
\node[vertex] (O) at (0,-0.5) { };
\node[vertex] (u) at (-1.2,0.6) { };
\node[vertex] (v) at (0,1) { };
\node[vertex] (w) at (1.2,0.6) { };
\draw (O) -- (u); \draw (O) -- (v); \draw (O) -- (w);
\node at (0,-1.0) {$G_2=K_{1,3}$};
\end{scope}
\end{tikzpicture}
\end{center}
\item[b.] If $G$ and $H$ are simple graphs with $L(G)$ isomorphic to $L(H)$, are $G$ and $H$ also isomorphic? Justify.
\item[c.] Prove that  if $G$ is a simple graph in which every vertex has degree $k$ (i.e.\ $G$ is $k$--regular), then every vertex of $L(G)$ has degree $2k-2$.
\end{itemize}
 
\solution{
\begin{itemize}
\item[i.] 
The line graph $L(G)$ has one vertex for each edge of $G$, and two vertices of $L(G)$ are adjacent iff the corresponding edges of $G$ share a common end-point.

\begin{itemize}
\item $G_1$ has three edges and each edge meets the other two. So $L(G_1)$ has three vertices, each adjacent to the other two; hence
\begin{center}
\begin{tikzpicture}[vertex/.style={draw,circle,inner sep=1.5pt,minimum size=12pt}, >=Stealth]
\node[vertex] (a) at (0,1) { };
\node[vertex] (b) at (-0.87,-0.5) { };
\node[vertex] (c) at (0.87,-0.5) { };
\draw (a) -- (b) -- (c) -- (a);
\node at (0,-1.0) {$L(G_1)\cong K_3$};
\end{tikzpicture}
\end{center}
\item 
$G_2$ has three edges all incident with the central vertex, so each pair of edges meet. Thus $L(G_2)$ also has three vertices all pairwise adjacent, so
\begin{center}
\begin{tikzpicture}[vertex/.style={draw,circle,inner sep=1.5pt,minimum size=12pt}, >=Stealth]
\node[vertex] (x) at (0,1) { };
\node[vertex] (y) at (-0.87,-0.5) { };
\node[vertex] (z) at (0.87,-0.5) { };
\draw (x) -- (y) -- (z) -- (x);
\node at (0,-1.0) {$L(G_2)\cong K_3$};
\end{tikzpicture}
\end{center}

\end{itemize}

\item[b.] In general \emph{no}. A simple counterexample is $G=K_{1,3}$ (the star on four vertices).  We saw above that $L(K_{1,3})\cong K_3$. Since $L(K_3)\cong K_3$ again, we have
\[
L(L(K_{1,3}))\cong K_3,
\]
but $K_{1,3}$ is not isomorphic to $K_3$ (they have different numbers of vertices and different degree sequences). Hence $G$ and $L(L(G))$ need not be isomorphic.

\item[c.]  A vertex of $L(G)$ corresponds to some edge $e=uv$ of $G$. In $L(G)$ the neighbours of the vertex corresponding to $e$ are exactly the edges of $G$ that share an end-point with $e$. The edges incident with the vertex $u$ in $G$ are $k$ in number, and similarly there are $k$ edges incident with $v$. Together these account for $k+k=2k$ incidences, but the edge $e$ itself is counted twice (once at $u$ and once at $v$), so the number of \emph{distinct} edges other than $e$ that meet $e$ is
\[
2k-2.
\]
Therefore the vertex of $L(G)$ corresponding to $e$ has degree $2k-2$. Since $e$ was arbitrary, every vertex of $L(G)$ has degree $2k-2$. \(\square\)
\end{itemize}
}
\end{questions}
\end{document}
